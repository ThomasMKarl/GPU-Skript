\documentclass[headsepline=3pt,headinclude=true,12pt,oneside]{scrartcl}
\usepackage[left=2cm,right=2cm,top=1cm,bottom=1cm,includeheadfoot]{geometry}
\usepackage{scrlayer-scrpage}
\usepackage{mwe}
\usepackage[origlayout=true,automark,colors={ph}]{URpagestyles}
\usepackage{lipsum} %fuer Fuelltexte

\usepackage{iftex}%automatische Auswahl des richtigen Fontloaders und der Eingabekodierung
%Es liefert das Makro \ifPDFTeX. Die Abfragen können entfernt werden, wenn nur eine bestimmte Variante verwendet wird.
\ifPDFTeX%falls mit pdfLaTeX kompiliert wird
	\usepackage[utf8]{inputenc}
	\usepackage[T1]{fontenc}
	\usepackage[ngerman]{babel} 
	%\usepackage[english]{babel} %Sparache aendern
\else%falls mit Lua oder XeLaTeX kompiliert wird
	\usepackage{fontspec}
	\usepackage{polyglossia}
	\setmainlanguage{german} 
	%\setmainlanguage{english} %Sprache aendern 
\fi
\usepackage{lmodern} %moderne Schriftarten
\usepackage{csquotes}
\usepackage{titletoc} %partielles Inhaltsverzeichnis

%alles mit Tabellen
\usepackage{array}
\usepackage{booktabs}
\usepackage{longtable}
\usepackage{csvsimple}
\usepackage{multirow}

%\usepackage[dvipsnames]{xcolor} %Farben, optionen fuer mehr Voreinstellungen
%aufpassen beim Drucken, Drucker sind fast alle schlecht kalibriert, erst eine Testseite drucken

\usepackage{xspace} %Abstaende
\usepackage{setspace} %Zeilenabstand

%alles mit Bildern
\usepackage{float}
\usepackage{graphicx}
\usepackage{caption}
\usepackage{subcaption}
\usepackage{sidecap}
\usepackage{wrapfig}
\usepackage{pdfpages}

%alles mit Mathematik
\DeclareMathSizes{18}{18}{18}{18}
\usepackage{upgreek}
\usepackage{paralist}
\usepackage{amsmath}
\usepackage{amssymb}
\usepackage{amscd}
\usepackage[output-decimal-marker={,}]{siunitx}
\usepackage{thmtools}
\usepackage{pgfplots}
\pgfplotsset{compat=1.16}

%code texen
\usepackage{listings}
\lstset{ %Beispielkonfiguration fuer C-code
  backgroundcolor=\color{lightgray!50!white},   % choose the background color; you must add \usepackage{color} or \usepackage{xcolor}; should come as last argument
  basicstyle=\footnotesize,        % the size of the fonts that are used for the code
  breakatwhitespace=false,         % sets if automatic breaks should only happen at whitespace
  breaklines=true,                 % sets automatic line breaking
  captionpos=b,                    % sets the caption-position to bottom
  commentstyle=\color{orange},     % comment style
  deletekeywords={sizeof},         % if you want to delete keywords from the given language
  extendedchars=true,              % lets you use non-ASCII characters; for 8-bits encodings only, does not work with UTF-8
  firstnumber=1,                   % start line enumeration with line 1000
  frame=single,                    % adds a frame around the code
  keepspaces=true,                 % keeps spaces in text, useful for keeping indentation of code (possibly needs columns=flexible)
  %keywordstyle=\color{red!50!black},% keyword style
  keywordstyle=\color{gray!50!black},% keyword style
  language=C++,                    % the language of the code
  morekeywords={__constant__, __shared__, __device__, __host__, __global__, __managed__, float2, float3, float4, dim3, int2, int3, int4, uint, half, cudaStream_t, cudaEvent_t, cudaError_t, cudnnError_t, cusparseError_t, cublasError_t, curandError_t, cufftError_t, cusolverError_t, __kernel, kernel, __constant, constant, __local, local, __private, private, __global, global, cl_device_id, cl_platform_id, cl_int, cl_uint, cl_float, cl_double, string, cl_mem, cl_kernel, size_t, cl_event, cl_program, cl_context, cl_command_queue, FILE, restrict, event_t, ptrdiff_t , intptr_t, uintptr_t, device_ptr, device_vector, host_vector, vector, Platform, Context, CommandQueue, Buffer, Program, Kernel, Sources, Device, cl_context_properties, cufftHandle, cufftComplex, cufftReal, cufftDouble, cufftDoubleComplex, clfftSetupData, clfftDim, fftw_complex, fftw_plan, curandStateSobol32_t, curandStateScrambledSobol32_t, curandStateSobol64_t, curandStateScrambledSobol64_t, curandState (XORWOW), curandStatePhilox4_32_10_t, curandStateMRG32k3a, curandStateMtgp32_t, curandGenerator_t, curandState, clrngMrg31k3pHostStream, clrngMrg31k3pStream, clprobdistExponential, clqmcLatticeRule, clqmcLatticeRuleStream, cublasHandle_t, cublasOperation_t, cusolverDnHandle_t, cusolverSpHandle_t, cusolverRfHandle_t, magma_queue_t, magma_int_t, LocalVector, CG, LocalStencil, __attribute__, aligned_allocator, auto},% if you want to add more keywords to the set
  numbers=left,                    % where to put the line-numbers; possible values are (none, left, right)
  numbersep=5pt,                   % how far the line-numbers are from the code
  numberstyle=\small\color{black}, % the style that is used for the line-numbers
  rulecolor=\color{black},         % if not set, the frame-color may be changed on line-breaks within not-black text (e.g. comments (green here))
  showspaces=false,                % show spaces everywhere adding particular underscores; it overrides 'showstringspaces'
  showstringspaces=false,          % underline spaces within strings only
  showtabs=false,                  % show tabs within strings adding particular underscores
  stepnumber=2,                    % the step between two line-numbers. If it's 1, each line will be numbered
  stringstyle=\color{green!50!black}, % string literal style
  identifierstyle=\color{gray!50!black},
  tabsize=2,	                   % sets default tabsize to 2 spaces
  title=\lstname                   % show the filename of files included with \lstinputlisting; also try caption instead of title
}
\expandafter\ifx\csname li\endcsname\relax
\let\li=\lstinline
\else\errmessage{li schon definiert \meaning\li}%
\fi


\setlength{\parindent}{0pt} %keine Absatzeinzüge
\setlength{\parskip}{\baselineskip}

%Links
\usepackage[colorlinks=true,
            linkcolor=black,
            urlcolor=UR@color@12!80!black,
            citecolor=OliveGreen]{hyperref}
\usepackage{tabularx} %Tabellencounter, nach hyperref

\renewcommand*{\familydefault}{\sfdefault}
\newcommand*{\pck}[1]{\texttt{#1}}
\newcommand*{\code}[1]{\texttt{#1}}
\newcommand*{\repl}[1]{\textrm{\textit{#1}}}
\newcommand{\cmd}[1]{\par\medskip\noindent\fbox{\ttfamily#1}\par\medskip\noindent}
\setcounter{secnumdepth}{\sectionnumdepth}
\newsavebox{\remarkbox}
\sbox{\remarkbox}{\emph{Anmerkung:~}}
\newcounter{iterator}
\usepackage{colortbl}

%meta Informationen
\author{Thomas Karl}
\title{rz412.ur.de:/temp/kat10110}
\subtitle{Dokumentation}
\date{\today}
\renewcommand{\titlepagestyle}{URtitle} 

\usepackage{lastpage}
\cfoot*{\thepage\ von \pageref{LastPage}}
\ofoot*{ }% the pagenumber in the center of the foot, also on plain pages
\ihead*{ }% Name and title beneath each other in the inner part of the foot
\ohead*{\leftmark}

\newcommand*\Laplace{\mathop{}\!\mathbin\bigtriangleup}

\begin{document}
	\maketitle
	
	Die Universität Regensburg bietet allen Studenten folgende Möglichkeiten zur Programmierung auf Grafikkarten:
	\begin{itemize}
    	\item Cip-Pool Chemie CH22.0.84 (Computernamen: CH201-CH210): Nvidia Quadro K600
	    \item Cip-Pool RZ4 (Computername: rz411): Nvidia Quadro K4000
    	\item Cip-Pool RZ4 (Computername: rz412): Nvidia Quadro P4000
	    \item Cip-Pool RZ4 (Computername: rz413): AMD Radeon RX 460 
	\end{itemize}
	
	Es handelt sich dabei um gewöhnliche CIP-Pool Rechner, die Anmeldung ist also wie gewohnt mittels RZ-Account möglich. Für den Fernzugriff auch von außerhalb kann ssh verwendet werden:
	
	\begin{center} \li`ssh <RZ-Kürzel>@<Computername>.ur.de` \end{center}
	
	Auf den Rechnern mit Nvidia Grafikkarten ist Version 9.1.85 des CUDA-Toolkits (beinhaltet OpenCL 1.2, im Moment fehlerhaft) installiert (unter \li`/usr/include/` und \li`/usr/lib/x86_64-linux-gnu/`). 
	Diese beinhaltet folgende Libraries:
	\begin{itemize}
		\item CUDA Runtime API
		\item CUDA Driver API
		\item CUDA Math API
		\item cuBLAS 
		\item NVBLAS 
		\item nvJPEG 
		\item cuFFT 
		\item nvGRAPH 
		\item cuRAND 
		\item cuSPARSE 
		\item NPP 
		\item NVRTC 
		\item THRUST 
		\item cuSOLVER
	\end{itemize}
	siehe \url{https://docs.nvidia.com/cuda/}
	
	Zudem stehen folgende Programme zur Verfügung:
	\begin{itemize}
		\item \li`nvcc` \\
		Nvidia CUDA Compiler \\
		\url{https://docs.nvidia.com/cuda/cuda-compiler-driver-nvcc/index.html}
		
		\item \li`nvprof` \\
		Nvidias Version des GNU Profilers \\
		\url{https://docs.nvidia.com/cuda/profiler-users-guide/index.html}	
		\item \li`nvlink`	  
		\item \li`nvdisasm`
		\item \li`cuobjdump` 
		\item \li`nvprune` \\
		\url{https://docs.nvidia.com/cuda/cuda-binary-utilities/index.html}
		\item \li`cuda-memcheck` \\
		Nvidias Valgrind Implementierung
		\url{https://docs.nvidia.com/cuda/cuda-memcheck/index.html}
	\end{itemize}
	
	Auf rz413 ist OpenCL in höheren Versionen verfügbar. In der Konsole lässt sich \li`clinfo` für mehr Informationen ausführen. 
		
	Alle Materialien des Kurses sowie nütliche Librarys auch für andere Gelegenheiten befinden sich im Directory \li`/temp/kat10110`. Dieses Directory ist von jedem Linux Rechner aus aufrufbar.
	Fast alle Librarys wurden für rz412 kompiliert und funktionieren nicht notwendigerweise auf anderen Rechnern. Lediglich die Bibliotheken unter \li`/temp/kat10110/ocl` wurden für rz413 kompiliert.
	\begin{itemize}
		\item \li`armadillo-9.200.6` \\ 
		ein High-Level C++ API für LAPACK \\
		\url{http://arma.sourceforge.net/}
		
		\item \li`googletest` \\
		ein Test Framework von Google in C++ \\
		\url{https://github.com/google/googletest}
		
		\item \li`gromacs` \\ 
		ein Programm mit zugehöriger Library für klassische Molekulardynamik, GPU Build \\
		\url{http://www.gromacs.org/GPU_acceleration}
		
		\item \li`ocl` \\
		Enthält OpenCL Librarys
		\begin{itemize}
			\item \li`clBLAS` \\  
			OpenCL Implementierung von BLAS \\
			\url{https://github.com/clMathLibraries/clBLAS}
			
			\item \li`clBLAST` \\ 
			verbesserte OpenCL Implementierung von BLAS \\
			\url{https://github.com/CNugteren/CLBlast}
			
			\item \li`clSPARSE` \\ 
			OpenCL Implementierung von cuSPARSE \\
			\url{https://github.com/clMathLibraries/clSPARSE}
			
			\item \li`clFFT` \\   
			OpenCL Implementierung von cuFFT \\
			\url{https://github.com/clMathLibraries/clFFT}
			
			\item \li`clRNG` \\    
			OpenCL Implementierung diverser RNGs \\
			\url{https://github.com/clMathLibraries/clRNG}
			
			\item \li`clProbDist` \\
			OpenCL Implementierung verschiedener Wahrscheinlichkeitsverteilungen \\
			\url{https://github.com/umontreal-simul/clProbDist}			
			
			\item \li`clQMC` \\  
			OpenCL Implementierung diverser Quasi-Monte-Carlo Algorithmen und quadi-random Verteilungen \\
			\url{https://github.com/umontreal-simul/clQMC} 
			
			\item \li`mwc64x` \\    
			extrem schmaler LCA RNG
\newpage			
			\item \li`ViennaCL-1.7.1` \\
			sparse Linear Algebra für OpenCL inkl. Python und MATLAB Bindings \\
			\url{http://viennacl.sourceforge.net/}
					
			\item \li`nengo-ocl` \\ 
			large-scale Brain Modelling \\
			\url{https://github.com/nengo/nengo-ocl}
			
			\item \li`ws` \\
			ein Programm zur Nullstellenberechnung komplexer Polynome
			\li`tex/doc\ger.pdf` enthält mehr Informationen.
			
			\item \li`nvidia_opencl_examples` \\
			Nvidia OpenCL SDK Code Samples, siehe \li`README.md` für mehr Informationen \\
			\url{https://developer.nvidia.com/opencl}
						
			\item \li`examples` \\
			enthält Minimalbeispiele für die OpenCL Math Librarys \\
			\li`compile.sh` kompiliert die Beispiele bzw, enthält eine Anleitung zum Kompilieren
			
			\item \li`lib` \\       
			enthält symbolische Links auf die Inhalte der lib Unterordner
			
			\item \li`include` \\  
			enthält symbolische Links auf die Inhalte der include Unterordner
		\end{itemize}
		
		\item \li`boost` \\ 
		Boost C++ Librarys in Version 1.66 \\
		\url{https://www.boost.org/}  
		
		\item \li`CUDA` \\
		enthält CUDA Librarys
		\begin{itemize} 		
			\item \li`MAGMA` \\
			eine LAPACK Implementierung für GPUs in C basierend auf cuBLAS und cuSPARSE \\
			\url{https://icl.cs.utk.edu/magma/}
			   
			\item \li`magmadnn` \\
			eine C++ Library für deep neural Networks basierend auf MAGMA und cuDNN \\
			\li`examples` enthält ein Tutorial in Form von einfachen Beispielen \\
			\url{https://magmadnn.bitbucket.io/docs/index.html}
			
			\item \li`AMGX` \\
			accelerated multi-grid für GPUs \\
			\url{https://developer.nvidia.com/amgx}
\newpage			
			\item \li`paralution` \\
			C++ Library für sparse iterative Solvers und Preconditioners \\
			\url{https://www.paralution.com/} 
			
			\item \li`SuiteSparse` \\ 
			Sparse Matrix Methoden inkl. MATLAB Interface \\
			\url{http://faculty.cse.tamu.edu/davis/suitesparse.html}
		
			\item \li`cuDNN` \\
			GPU beschleunigte Primitives für neuronale Netze (v7.5), Voraussetzung für die meisten Machine-Learning Frameworks (Caffee, Tensorflow...) \\
			\li`cudnn_samples_v7` enthält die offiziellen Beispiele für cuDNN \\
			\url{https://developer.nvidia.com/cudnn}
			
			\item \li`TensorRT` \\
			Optimierer und Runtime für kompilierte Machine-Learning Modelle verschiedener Frameworks (v5.1), optionale Voraussetzung für Tensorflow \\
			\url{https://developer.nvidia.com/tensorrt}
			 
			\item \li`SilverLining` \\  
			GPU beschleunigtes SDK für 3D-gerenderte Himmels- und Wolkeneffekte \\
			\url{https://sundog-soft.com/}
			
			\item \li`triton-sdk-eval` \\	
			GPU beschleunigtes SDK für 3D-gerenderte Wassereffekte \\
			\url{https://sundog-soft.com/features/ocean-and-water-rendering-with-triton/}
			  
			\item \li`quda-0.9.0` \\  
			Library für QCD auf GPUs \\
			\url{http://lattice.github.io/quda/}

			\item \li`cuda-samples` \\
			enthält die offiziellen Beispiele für CUDA in Version 9.2 \\
			\url{https://docs.nvidia.com/cuda/archive/9.2/}
		
			\item \li`lib` \\       
			enthält symbolische Links auf die Inhalte der lib Unterordner
			
			\item \li`include` \\  
			enthält symbolische Links auf die Inhalte der include Unterordner   
		\end{itemize}
		
		\item \li`gpu-kurs` 		
		die jeweils erste Zeile eines Programms enthält Anweisungen zum Kompilieren, mehr Informationen im Skript
		\begin{itemize} 
		    \item \li`gpuerror.h` \\
		    nach Inkludieren aktiviert \li`-D DEBUG_CUDA` (\li`-D DEBUG_OCL -lOpenCL`) beim Kompilieren einige nützliche Funktionen
		    
			\item \li`cluster.cu` \\
			Ein einfaches Beispiel zur Simulation eines GPU-Clusters mit MPI gesteuerten Prozessen und verschiedenen Grafikkarten. Das Programm wird wie gewohnt kompiliert und dabei mit \textit{libmpi} gelinkt. Danach muss das Programm mittels Hostfile (\li`mpirun -H ...`)auf verschiedenen Computern (z.B. Linux CIP-Pool Chemie) ausgeführt werden. Jeder MPI Prozess steuert seinen Prozessor und dieser genau eine Grafikkarte. Verschiedene Kerne können zusätzlich mit OpenMP benutzt werden.
			
			\item \li`conv.cu` \\
			ein convulutional neural Network mit cuDNN

			\item \li`cudnn.cu` \\
			einfaches Image-Processing mit cuDNN
			
			\item \li`curand.cu` \\
			Anwendungsbeispiel für die Erzeugung von Zufallszahlen auf Host oder Device mit cuRAND			
			
			\item \li`device_test.cu` \\ 
			eine Funktion zur Abfrage der Eigenschaften vorhandener Nvidia Grafikkarten (\li`clinfo` für OpenCL)			
			
			\item \li`heat.cu` \\
			Lösung einer zeitabhängigen eindimensionalen Wärmeleitungsgleichung mit Jacobi-Iteration
			
			\item \li`magmadnn.cu` \\ 
			ein convulutional neural Network mit magmaDNN		
			
			\item \li`nbody.cu` \\
			Verlet Algorithmus
			
			\item \li`pi.cu` \\
			Berechnung von $\pi$ mittel Monte-Carlo Integration und Thrust Vektoren
			
			\item \li`reduction.cu` \\  
			Reduktionen in fünf verschiedenen Optimierungen			
			
			\item \li`reverse_shared.cu` \\ 
			Illustration von shared Memory durch Revertieren eines Arrays	
			
			\item \li`thrust.cu` \\
			einfache Thrust Beispiele (STL für CUDA)			
			
			\item \li`LA_examples.pdf` \\ 
			kommentierte Code Beispiele für cuBLAS, cuSPARSE und MAGMA \\
			\url{https://developer.nvidia.com/sites/default/files/akamai/cuda/files/Misc/mygpu.pdf}
\newpage			
			\item \li`fft_test` 
			\begin{itemize}
				\item \li`clfft_test.cu` \\
				ein Vergleich von FFTW, cuFFT und clFFT				
				
				\item \li`output.dat` \\
				enthält Ausgabe Daten von \li`clfft_test.cu`
			\end{itemize}
			
			\item \li`haxpy`
			\begin{itemize}
				\item \li`fp16_conversion.h` \\
				ein Header von Nvidia für Arithmetik mit 16 Bit Präzision
				
				\item \li`haxpy.cu` \\
				ein Beispiel zur Verwendung des (2 Byte) Half Datentyps und Vergleich mit Float und Double
				
				\item \li`prec.dat` \\
				enthält Ausgabe Daten von \li`haxpy.cu`
			\end{itemize}
			
			\item \li`linalg`
			\begin{itemize}
				\item \li`lu_cusolver.cu` \\
				LU Zerlegung mit cuSOLVER
				
				\item \li`lu_magma_cpu.c` \\
				LU Zerlegung mit MAGMA (CPU Interface)
				 
				\item \li`lu_magma_gpu.c` \\
				LU Zerlegung mit MAGMA (GPU Interface)
				
				\item \li`magma.c` \\
				Reduktion mit MAGMA
				
				\item \li`matmul.cu` \\
				Matrixmultiplikation mit und ohne shared Memory
				
				\item \li`para.cpp` \\
				iterative Lösung einer Laplace Gleicung mit Paralution
			\end{itemize}
			
			\item \li`vecadd`
			\begin{itemize}
				\item \li`cl_vecadd.c`  
				Host Code der Vektoraddition mit OpenCL (C API) 
			
				\item \li`cl_vecadd.cl`  
				Device Code der Vektoraddition mit OpenCL
			 
				\item \li`cl_vecadd.cpp`  
				Host Code der Vektoraddition mit OpenCL (C++ API) 
			
				\item \li`vecadd_const.cu` 
				Vektoraddition in CUDA mit constant Memory
			
				\item \li`vecadd.cu` \\
				Vektoraddition in CUDA
			\end{itemize}
			
			\textcolor{red}{Bei folgenden Ordnern handelt es sich um Lehrmaterialien von Nvidia, die nicht öffentlich einsehbar und damit nicht zur Veröffentlichung bestimmt sind.}
			
			\item \li`AC_teaching_kit` \\
			Nvidias CUDA Kurs mit Vorlesungsfolien, Übungen, Projekten und einem offiziellen E-Book
			
			\item \li`DLI_teaching_kit` \\
			Nvidias deep Learning Kurs mit Vorlesungsfolien und Übungen
			
			\item \li`Isaac_teaching_kit` \\
			Nvidias Robotik Teaching Kit für Jetson Produkte \\
			\url{https://docs.nvidia.com/isaac/isaac/doc/index.html}
			
			\item \li`Skript<Version>.pdf` \\
			das jeweils aktuelle Skript zum Kurs
			
			\item \li`Aufgaben<Version>.pdf` \\
			die jeweils aktuelle Aufgabensammlung zum Kurs
			
			\item \li`libdoc` \\
			dieses Dokument
		\end{itemize}  
		
		\item \li`hdf5` \\
		Library für das Lesen und Schreiben von HDF5 Dateien, Python Bindings sind Voraussetzung für das Exportieren von Tensorflow Modellen mit Keras \\
		\url{https://www.hdfgroup.org/downloads/hdf5}
		  
		\item \li`machine_learning` 		
		\begin{itemize} 
			\item \li`get-pip.py` \\
			das Ausführen mit python2 (python3) installiert pip2 (pip3), einen Paketmanager für Python, lokal im Home-Directory \\
			kann z.B. zur lokalen Installation von Tensorflow genutzt werden \\
			\url{https://www.tensorflow.org/install/pip}
			
			\item \li`liblinear` \\  
			Library für large linear Classification, stellt Bindings für die relevantesten Programmiersprachen zur Verfügung \\
			\url{https://www.csie.ntu.edu.tw/~cjlin/liblinear/}
			
			\item \li`libsvm` \\
			enthält in \li`bin` Programme für Support Vector Machines (Klassifizierung oder Lineare Regression), stellt Bindings für die relevantesten Programmiersprachen zur Verfügung \\
			\url{https://www.csie.ntu.edu.tw/~cjlin/libsvm/}			
			
			\item \li`mlpack-3.0.4` \\
			C++ Framework für klassisches machine-Learning (z.B. Clustering, PCA...) inkl. Python Bindings und Kommandozeilen Programme basierend auf Armadillo \\
			\url{https://www.mlpack.org/}
			
			\item \li`opencv-4.1.0` \\
			state-of-the-art Library für Bildbearbeitung mit klassischen und machine-Learning Algorithmen, parallelisiert mit CUDA \\
			\url{https://opencv.org/about/}
			
			\item \li`vigra` \\
			generic Template Library in C++ für multidimensionale Bilddateien \\
			\url{https://ukoethe.github.io/vigra/}
		\end{itemize}  
		 
		\item \li`OpenBLAS` \\
		eine optimierte (nicht parallelisierte) BLAS Library in C, Voraussetzung für diverse Bibliotheken (z.B. MAGMA) \\
		\url{https://www.openblas.net/}
	\end{itemize}						
\end{document}