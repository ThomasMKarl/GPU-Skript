	\chapter{HPC Librarys}	
	Um die Arbeit zu erleichtern, wurden sowohl für OpenCL als auch für CUDA umfangreiche High Performance Bibliotheken implementiert. Beteiligt sind hier die Entwickler von CUDA und OpenCL, eine große Community sowie Forschungsgruppen an Universitäten und privaten Forschungseinrichtungen. Die wichtigsten Themengebiete sind lineare Algebra, Machine Learning, Fast Fourier Transformationen und Monte Carlo Algorithmen, da diese die Grundlage für die meisten Algorithmen in IT und Naturwissenschaften bilden. Die folgenden Kapitel behandeln Einführungen in die Funktionsweise der wichtigsten Bibliotheken für CUDA und OpenCL. Aufgrund des großen Umfangs kann natürlich nur ein kleiner Teil davon gezeigt werden. Es ist daher unumgänglich, in größeren Projekten direkt mit dem Quellmaterial zu arbeiten, also mit den Dokumentationen und Programming Guides der Entwickler.
	\newpage	
	
	\startcontents[THRUST]
   	\printcontents[THRUST]{l}{2}{\section*{THRUST}\setcounter{tocdepth}{2}}
		\section{THRUST: STL f\"ur GPUs}
	THRUST ist die CUDA-Implementierung der Standard Template Library für C++ und ist Teil des CUDA-Toolkits. THRUST orientiert sich an C++11 und beseht aus vier Teilen.
		
		\subsection{Container}
		Grundlage für THRUST bilden die Container-Klassen zur Erzeugung dynamischer Datenstrukturen. Diese Klassen befinden sich im Namespace \li`thrust`. Die Klasse \li`host_vector` entspricht dabei dem C++ Vektor. Mit \li`device_vector` existiert eine Vektorklasse, die sich ebenfalls über Member-Funktionen steuern lässt, aber zu jedem Zeitpunkt im Speicher des Device liegt.
		\begin{lstlisting}[caption=THRUST Vektoren]
		#include <thrust/host_vector.h>
		#include <thrust/device_vector.h>

		...

		thrust::host_vector<int> H(4);

		H[0] = 14;
		H[1] = 20;
		H[2] = 38;
		H[3] = 46;

		std::cout << "H has size " << H.size() << std::endl;

		H.resize(2);

		thrust::device_vector<int> D = H;

		D[0] = 99;
		D[1] = 88;
		\end{lstlisting}
		
		In diesem Beispiel wird ein Vektor von Ganzzahlen erstellt und dessen Elemente gesetzt. Die Größe kann über die getter-Funktion \li`size` ausgegeben werden. Die Größe kann dynamisch verringert oder vergrößert werden. Danach wird der Hostvektor explizit in einen Devicevektor kopiert. Der Devicevektor lässt sich dabei auch vom Host modifizieren.
		
		Die Idee hinter Devicevektoren ist, Member-Funktionen zu benutzen, die auf diesen Datenstrukturen operieren und dabei in CUDA parallelisiert sind. Devicevektoren werden wie gewöhnliche Vektoren vom Host gesteuert, und können nicht direkt im Device Code verwendet werden, da sich im Device Speicher nicht dynamisch verwalten lässt.
		
		\subsection{Iteratoren}
		Zur besseren Steuerung der Vektoren bietet THRUST Iteratoren an.
		\begin{lstlisting}[caption=THRUST Iteratoren]
		thrust::device_vector<int> D(10, 1);
		thrust::fill(D.begin(), D.begin() + 7, 9);
		
		thrust::host_vector<int> H(D.begin(), D.begin()+5);
		thrust::sequence(H.begin(), H.end());
		
		thrust::copy(H.begin(), H.end(), D.begin());
		\end{lstlisting}
		
		\li`begin` und \li`end` sind Iteratoren, die auf Anfang und Ende des belegten Speichers zeigen. So können THRUST Funktionen benutzt werden, um Vektoren zu füllen oder zu Kopieren. Die erste Funktion füllt einen Vektor von Element 0 bis 6	mit 9. Dann wird ein Vektor erstellt als Kopie der ersten fünf Elemente des vorherigen. \li`sequenze` füllt den Vektor mit aufsteigenden Ganzzahlen. Im letzten Schritt wird ein Vektor von Anfang bis Ende An den Beginn eines anderen Vektors kopiert.
		
		Weitere Iteratoren sind
		\begin{itemize}
		\item \li`constant_iterator`
		\item \li`counting_iterator`
		\item \li`transform_iterator`
		\item \li`permutation_iterator`
		\item \li`zip_iterator`
		\end{itemize}
		und können in Kapitel 4 des Qick Start Guides nachgeschlagen werden. \autocite{thrustQSG}
		
		\subsection{Funktoren}\label{funk}
		Funktoren sind spezielle Datentypen, die Funktionen beinhalten, welche bestimmte Operationen ausführen. Diese Operationen sollen Elementweise auf einen Vektor angewendet werden. Folgendes Beispiel definiert einen Funktor für die Saxpy Operation für Host und Device:
		\begin{lstlisting}[caption=THRUST Funktoren]
		struct saxpy_functor
		{
			const float a;

			saxpy_functor(float _a) : a(_a) {}

			__host__ __device__
			float operator()(const float& x, const float& y) const {return a * x + y;}
		};
		\end{lstlisting}
		
		Nun kann mittels \li`thrust::transform(X.begin(), X.end(), Y.begin(), Y.begin(), saxpy_functor(A))` diese Operation auf jedes Element von \li`X` und \li`Y` angewendet und in \li`Y` gespeichert werden (siehe \ref{algo}).	Es existieren vordefinierte Funktoren wie \li`multiplies` oder \li`plus`.
		
		\subsection{Algorithmen}\label{algo}
		THRUST implementiert schnelle parallele Algorithmen in CUDA für dessen Containerklassen. Neben den genannten Transformationen (siehe \ref{funk}) existiert eine Implementierung eines parallelen Radix-Sortierverfahrens.
		\begin{lstlisting}[caption=THRUST Sortieren]	
		#include <thrust/sort.h>
		#include <thrust/functional.h>
		
		...
		
		const int N = 6;
		
		int A[N] = {1, 4, 2, 8, 5, 7};
		thrust::sort(A, A + N);
		thrust::sort(A, A + N, thrust::greater<int>());


		int    keys[N] = {  1,   4,   2,   8,   5,   7};
		char values[N] = {'a', 'b', 'c', 'd', 'e', 'f'};
		thrust::sort_by_key(keys, keys + N, values);
		\end{lstlisting}
		
		\li`sort` sortiert einen Vektor numerisch in aufsteigender Reihenfolge (inplace). Dies lässt sich mit einem Funktor kombinieren, um z.B. absteigend zu sortieren. \li`sort_by_keys` sortiert ein Array anhand von Eigenschaften eines anderen Arrays, den sogenannten Keys. In diesem Fall wird jedem Buchstaben eine Zahl zugeordnet und die Liste nach diesen Zahlen sortiert. Es existieren ebenfalls \li`stable_sort` uns \li`stabel_sort_by_keys`, die die relative Ordnung von Elementen mit gleichen Werten, die zur Ordnung klassifizieren, erhalten. Wenn also a und b den gleichen Key erhalten, ist so garantiert, dass a und b nicht vertauscht werden.
		
		\newpage
		Extrem nützlich für GPUs sind die Reduktionen von THRUST.
		\begin{lstlisting}[caption=THRUST Reduktionen]
		int sum = thrust::reduce(D.begin(), D.end(), (int) 0, thrust::plus<int>());
		
		#include <thrust/count.h>
		int result = thrust::count(vec.begin(), vec.end(), 1);
		\end{lstlisting}
		
		Erstere Funktion vollzieht eine Summenreduktion, Letztere zählt den Wert eins in einem Vektor. Zum Suchen bestimmter Werte implementiert THUST eine parallele Version einer Binärsuche. Eine Liste aller Reduktionen befindet sich unter \url{http://thrust.github.io/doc/group__reductions.html} in der THRUST Dokumentation. \autocite{thrustDoc}
		
		Mittels \li`transform_reduce` lassen sich Reduktionen mit selbstgeschriebenen Funktoren kombinieren. Soll eine Reduktion von THRUST auf ein Array angewendet werden, dass von einem \Gls{Kernel} modifiziert wurde oder an eines übergeben werden soll, eignet sich die Klasse \li`device_ptr`.		
		\begin{lstlisting}[caption=THRUST Device Pointer]
		size_t N = ...;

		int *raw_ptr;
		cudaMalloc((void **) &raw_ptr, N * sizeof(int))

		thrust::device_ptr<int> dev_ptr(raw_ptr);
		thrust::fill(dev_ptr, dev_ptr + N, (int) 0);

		thrust::device_ptr<int> dev_ptr = thrust::device_malloc<int>(N);
		raw_ptr = thrust::raw_pointer_cast(dev_ptr);
		\end{lstlisting}	
			
		Im ersten Beispiel wird aus einem gewöhnlichen Devicevektor ein \li`device_ptr` Objekt erstellt, im zweiten ein Pointer extrahiert. Auf dieses Objekt kann ein gewöhnlicher THRUST Algorithmus angewendet werden.
		
		Ferner existieren Scan\footnote{\url{http://thrust.github.io/doc/group__prefixsums.html}}- und Reordering\footnote{\url{http://thrust.github.io/doc/group__reordering.html}} Algorithmen.		
	\stopcontents[THRUST]
		
	\newpage
	\startcontents[Random Number Generators]
   	\printcontents[Random Number Generators]{l}{2}{\section*{Random Number Generators}\setcounter{tocdepth}{2}}	
		\section{Random Number Generators}
		\subsection{cuRAND}
		cuRAND ist eine Bibliothek zur Generierung von Pseudo- und Quasi-Zufallszahlen. Sie stellt ein \Gls{API} sowohl für Device als auch für Host zur Verfügung. Zur Benutzung inkludiert man den Header \textit{curand.h} für Host und \textit{curand{\_}kernel.h} für Device. Zusätzlich muss mit \textit{libcurand} gelinkt werden. Um gegen die statische Version zu linken wir folgendes Kommando empfohlen: 
		
		\li`g++ myCurandApp.c -lcurand_static -lculibos -lcudart_static -lpthread -ldl`
		
			\subsubsection{Pseudo Random}
			Tabelle \ref{tab5:prng} zeigt die implementierten RNGs. Tabelle \ref{tab5:qrng} zeigt alle exisitierenden Verteilungen.
			\begin{table}[h]
			\centering
			\begin{tabular}{|l|l|}
				\hline
				\textbf{Generator} & \textbf{Beschreibung} \\ \hline\hline
				CURAND{\_}RNG{\_}PSEUDO{\_}XORWOW    & XOR-shift Algorithmus \\ \hline
				CURAND{\_}RNG{\_}PSEUDO{\_}MRG32K3A  & Combined Multiple Recursive Algorithmus \\ \hline
				CURAND{\_}RNG{\_}PSEUDO{\_}MT19937   & Mersenne Twister, CPU optimiert \\
				                                     & verändertes Ordering, nur Host API, SM $\geq$ 3.5 \\ \hline
				CURAND{\_}RNG{\_}PSEUDO{\_}MTGP32    & Mersenne Twister \\
				                                     & GPU optimierte Parameter \\ \hline
				CURAND{\_}RNG{\_}PHILOX4{\_}32{\_}10 & non-cryptographic Counter Based Algorithmus \\ \hline\bottomrule
			\end{tabular}
			\caption{Liste der Pseudo RNGs}
			\label{tab5:prng}
			\end{table}
			
			Zur Benutzung in der Host \Gls{API} muss zunächst ein Buffer erstellt werden. Dann wird ein Generator ausgewählt und mit einem bestimmten Seed initialisiert. Die selben Seeds produzieren die selben Sequenzen. Anschlie\ss end werden die Zufallszahlen anhand einer bestimmten Verteilung erzeugt. Tabelle \ref{tab5:dist} zeigt alle möglichen Verteilungen. Der Seed kann ebenfalls als Zufallszahl gesetzt werden, beispielsweise mit einem Speicherrauschen.
			\begin{lstlisting}[caption=cuRAND: Host API]
			uint n = ...;			
			
			float *data;
			cudaMalloc((void **)&devData, n*sizeof(float));
			
			curandGenerator_t gen;
			curandCreateGenerator(&gen, CURAND_RNG_PSEUDO_DEFAULT);
			long long seed = ...;
			curandSetPseudoRandomGeneratorSeed(gen, seed);
			curandGenerateUniform(gen, devData, n);
			
			...
			
			curandDestroyGenerator(gen);
			cudaFree(devData);
			\end{lstlisting}	
					
			Die Zufallszahlen können auf den Host zurückkopiert, in einem anderen \Gls{API} verwendet oder in einem eigenen \Gls{Kernel} benutzt werden.
			
			\begin{table}[h]
			\centering
			\begin{tabular}{|l|l|}
				\hline
				\textbf{Verteilung} & \textbf{Beschreibung} \\ \hline\hline
				\li`curandGenerate(...)` & 32bit Zufallszahlen, jedes Bit zufällig \\ \hline
				\li`curandGenerateLongLong(...)` & 64bit Zufallszahlen, jedes Bit zufällig \\ \hline
				\li`curandGenerateUniform(...)` & Uniforme Zufallszahlen im Bereich $(0,1]$ \\ \hline
				\li`curandGenerateNormal(..., float m, float s)` & \multirowcell{2}{Normalverteilte Zufallszahlen mit \\ Mittelwert und Standardabweichung} \\
				 & \\ \hline
				\li`curandGenerateLogNormal(..., float mean, float stddv)` & Log-Normalverteilte Zufallszahlen \\ \hline
				\li`curandGeneratePoisson(..., size_t n, double lambda)` & \multirowcell{2}{Ganzzahlen der Verteilung (für $\lambda > 0$) \\ $P(k)_{\lambda} = \lambda^k/k!\cdot\mathrm{e}^{-\lambda}$, $k=0,1,2...$} \\
				 & \\ \hline
				\li`curandGenerateUniformDouble(...)` & Doppelte Präzision \\ \hline
				\li`curandGenerateNormalDouble(..., float mean, float stddv)` & Doppelte Präzision \\ \hline
				\li`curandGenerateLogNormalDouble(..., float mean, float stddv)` & Doppelte Präzision \\ \hline\bottomrule
			\end{tabular}
			\caption{Liste der Pseudo RNGs}
			\label{tab5:dist}
			\end{table}
			
			Zur Benutzung des Device \Glspl{API} muss ein sogenannter State für jeden \Gls{Thread} erstellt werden. Dieser wird in einem eigenen \Gls{Kernel} erstellt.
			Dazu muss die Funktion \li`curand_init` gerufen werden. Dieser wird ein Seed, eine Sequenznummer und ein Offset übergeben. Der Seed kann auf dem Host zufällig initialisiert werden. Die Sequenznummer hängt vom \Gls{Thread}index ab und ist damit für jeden \Gls{Thread} verschieden.
			\begin{lstlisting}[caption=cuRAND: Device API States]
			__global__ void setup_kernel(curandState *state)
			{
				uint id = threadIdx.x + blockIdx.x * blockDim.x;
				curand_init(1234, id, 0, &state[id]);
			}
			
			...
			
			uint n = ...;
			curandState *devStates;
			cudaMalloc(&devStates, n*sizeof(curandState));
			setup_kernel<<<...,...>>>(devStates);		
			\end{lstlisting}
			
			Nun kann in einem separaten \Gls{Kernel} anhand dieses States eine Verteilung benutzt werden. So muss pro Programm und \Gls{Thread} die langsame \li`curand_init` Funktion nur einmal gerufen werden. Um die Performance weiter zu verbessern, wird jeder State erst in den \gls{local Memory} kopiert und am Ende des \Glspl{Kernel} zurückkopiert. Werden mehrere Zufallszahlen benötigt, wird so der State in den Cache geladen.  \\ \\
			\begin{lstlisting}[caption=cuRAND: Device API Generierung]
			__global__ void generate_kernel(curandState *state, uint *result)
			{
				uint id = threadIdx.x + blockIdx.x * blockDim.x;

				curandState localState = state[id];

				result[id] = curand(&localState);
				
				state[id] = localState;
			}
			
			...
			
			uint *devResults;
			cudaMalloc(&devResults, n*sizeof(uint));
			generate_kernel<<<...,...>>>(curandState *state, uint *devResults);
			\end{lstlisting}
			
			Benötigt man einen anderen Generator, muss lediglich ein anderer State erzeugt werden:
			\begin{itemize}
			\item \li`curandState (XORWOW)`
			\item \li`curandStatePhilox4_32_10_t`
			\item \li`curandStateMRG32k3a`
			\item \li`curandStateMtgp32_t`
			\end{itemize}
			
			In Kapitel 3.1.4 der cuRAND Dokumentation befindet sich eine Liste aller Generatoren. \autocite{curandDoc}
					
			\subsubsection{Quasi Random} 
			Neben den gewöhnlichen Pseudo Zufallszahlen werden des Öfteren sogenannte Quasi Zufallszahlen verwendet. Dies bezeichnet Sequenzen von Zahlen die im mathematischen Sinn eine geringe Diskrepanz aufweisen. Hauptanwendungsgebiet ist Monte-Carlo Integration, da Quasi-Monte-Carlo Integration unter bestimmten Umständen eine höhere Konvergenzgeschwindigkeit aufweist. Die bekannteste und schnellste ist die Sobol Sequenz. Zur Benutzung in der Host \Gls{API} muss lediglich ein entsprechender Generator aus Tabelle \ref{tab5:qrng} verwendet werden.
			\begin{table}[h]
			\centering
			\begin{tabular}{|l|l|}
				\hline
				\textbf{Generator} & \textbf{Beschreibung} \\ \hline\hline
				CURAND\_ RNG\_ QUASI\_ SOBOL32     & 32bit Sobol Sequenz \\ \hline
				CURAND\_ RNG\_ QUASI\_ SCRAMBLED\_ SOBOL32 & Scrambled 32bit Sobol Sequenz \\ \hline
				CURAND\_ RNG\_ QUASI\_ SOBOL64 & 64bit Sobol Sequenz \\ \hline
				CURAND\_ RNG\_ QUASI\_ SCRAMBLED\_ SOBOL64 & Scrambled 64bit Sobol Sequenz \\ \hline\bottomrule
			\end{tabular}
			\caption{Liste der Quasi RNGs}
			\label{tab5:qrng}
			\end{table}
			
			Zur Benutzung in der Device \Gls{API} muss ein entsprechender State benutzt werden:
			\begin{itemize}
			\item \li`curandStateSobol32_t`
			\item \li`curandStateScrambledSobol32_t`
			\item \li`curandStateSobol64_t`
			\item \li`curandStateScrambledSobol64_t`
			\end{itemize}
			
			In Kapitel 3.2 der cuRAND Dokumentation befindet sich eine Liste der entsprechenden Verteilungen. \autocite{curandDoc}

			\newpage						
						
		\subsection{Random mit OpenCL}
			\subsubsection{clRNG}
			\url{http://clmathlibraries.github.io/clRNG/htmldocs/index.html}
			
			\url{https://github.com/clMathLibraries/clRNG}			
			\begin{lstlisting}[caption=clRNG Beispiel]
			#define CLRNG_SINGLE_PRECISION                                  
			#include <clRNG/mrg31k3p.clh>                                  
                                                                
			__kernel void example(__global clrngMrg31k3pHostStream *streams, 
				__global float *out)
			{                                                               
				int gid = get_global_id(0);                                 
                                                                      
				clrngMrg31k3pStream workItemStream;                          
				clrngMrg31k3pCopyOverStreamsFromGlobal(1, &workItemStream, 
					&streams[gid]); 
                                                                   
				out[gid] = clrngMrg31k3pRandomU01(&workItemStream);       
			}                                                                                                                               
    
			//////////////////////////////////////////////////////////////////////
			
			cl_int err, streamBufferSize, numWorkItems = ...;

			clrngMrg31k3pStream *streams = clrngMrg31k3pCreateStreams(NULL, 
				numWorkItems, &streamBufferSize, (clrngStatus *)&err);

			cl_mem bufIn = clCreateBuffer(ctx, CL_MEM_READ_ONLY|CL_MEM_COPY_HOST_PTR, 
				streamBufferSize, streams, &err);
			cl_mem bufOut = clCreateBuffer(ctx, CL_MEM_WRITE_ONLY | 
				CL_MEM_HOST_READ_ONLY, numWorkItems * sizeof(cl_float), NULL, &err);
			clSetKernelArg(kernel, 0, sizeof(bufIn),  &bufIn);
			clSetKernelArg(kernel, 1, sizeof(bufOut), &bufOut);

			clEnqueueNDRangeKernel(queue, kernel, 1, NULL, &numWorkItems, 
				NULL, 0, NULL, NULL);
			\end{lstlisting}
			
			\subsubsection{clProbDist}
			\url{http://umontreal-simul.github.io/clProbDist/htmldocs/index.html}
			
			\url{https://github.com/umontreal-simul/clProbDist}
			\begin{lstlisting}[caption=clProbDist Beispiel]
			#include <clProbDist/exponential.clh>                             
                                                                          
			__kernel void example(__global const clprobdistExponential *dist, 
				__global double *out)                       
			{                                                                 
				int gid = get_global_id(0);                                   
				int gsize = get_global_size(0);                               
				
				double quantile = (gid + 0.5) / gsize;                        
				out[gid] = clprobdistExponentialInverseCDFWithObject(dist, 
					quantile, (void *)0); 
			} 
			 
			//////////////////////////////////////////////////////////////////////
			
			cl_int err, distBufferSize, numWorkItems = ...;
			clprobdistExponential *dist;
			dist = clprobdistExponentialCreate(1.0, &distBufferSize, 
				(clprobdistStatus *)&err);    
			
			cl_mem bufIn = clCreateBuffer(ctx, CL_MEM_READ_ONLY|CL_MEM_COPY_HOST_PTR, 
				distBufferSize, dist, &err);
			cl_mem bufOut = clCreateBuffer(ctx, CL_MEM_WRITE_ONLY | 
				CL_MEM_HOST_READ_ONLY, numWorkItems * sizeof(cl_double), NULL, &err);
			clSetKernelArg(kernel, 0, sizeof(bufIn),  &bufIn);
			clSetKernelArg(kernel, 1, sizeof(bufOut), &bufOut);
			
			clEnqueueNDRangeKernel(queue, kernel, 1, NULL, &numWorkItems, 
				NULL, 0, NULL, NULL);		
			\end{lstlisting}
			
			\subsubsection{clQMC}
			\url{http://umontreal-simul.github.io/clQMC/htmldocs/index.html}
			
			\url{https://github.com/umontreal-simul/clQMC}
			\begin{lstlisting}[caption=clQMC Beispiel]    
			#define CLQMC_SINGLE_PRECISION                                 
			#include <clQMC/latticerule.clh>                               
                                                                       
			__kernel void example(__global const clqmcLatticeRule *lat, 
				__global float *out)                     
			{                                                              
				int gid = get_global_id(0);                                
				int gsize = get_global_size(0);                            
				int dim = clqmcLatticeRuleDimension(lat);                  
                                                                       
				clqmcLatticeRuleStream stream;                             
				clqmcLatticeRuleCreateOverStream(&stream, lat, gsize, gid, (void*)0);  
				for (int j = 0; j < dim; j++) 
					out[j * gsize + gid] = clqmcLatticeRuleNextCoordinate(&stream); 
			}
			
			cl_int err, latBufferSize, numWorkItems = ...;
			clqmcLatticeRule *lat;               
			lat = clqmcLatticeRuleCreate(numWorkItems, 3, (cl_int[]){1, 27, 15}, 
				&latBufferSize, (clqmcStatus *)&err);  

			cl_mem bufIn = clCreateBuffer(ctx, CL_MEM_READ_ONLY|CL_MEM_COPY_HOST_PTR, 
				latBufferSize, lat, &err);
			cl_mem bufOut = clCreateBuffer(ctx, CL_MEM_WRITE_ONLY | 
				CL_MEM_HOST_READ_ONLY, numWorkItems*clqmcLatticeRuleDimension(lat) 
				*sizeof(cl_float), NULL, &err);
			clSetKernelArg(kernel, 0, sizeof(bufIn),  &bufIn);
			clSetKernelArg(kernel, 1, sizeof(bufOut), &bufOut);

			clEnqueueNDRangeKernel(queue, kernel, 1, NULL, &numWorkItems, 
				NULL, 0, NULL, NULL);
			\end{lstlisting}
				
	\stopcontents[Random Number Generators]
		
	\newpage
	\startcontents[Fast Fourier Transformation]
   	\printcontents[Fast Fourier Transformation]{l}{2}{\section*{Fast Fourier Transformation}\setcounter{tocdepth}{2}}
		\section{Fast Fourier Transformation}
		\subsection{Theorie}
		Aus Wikipedia \autocite{wikiFFT}:
		
		Die Fouriertransformation (FT) ist die wohl wichtigste mathematische Transformation im Bereich Daten- und Signalverarbeitung. Sie findet Anwendung in der Audio- und Videobearbeitung, im maschinellen Lernen und in physikalischen und chemischen Simulationen. Da man in Simulationen nur auf diskrete Werte zugreifen kann, wird in solchen üblicherweise die diskrete Form der FT verwendet (DFT):
		
		\begin{equation}
		f_m = \sum^n_{k=0}x_k\cdot\mathrm{e}^{\left(-\frac{2\pi i}{2n}mk\right)}\qquad m=0,...,n-1
		\end{equation}
		
		Wobei $f_m$ das $m$-te Element der Fouriertransformation eines Vektors ($x_0$, ..., $x_{n-1}$) darstellt. Die meisten Algorithmen, die eine DFT implementieren, verwenden die sogenannte Fast Fourier Transformation (FFT): Teilt man den Vektor in gerade $x^{\prime}_k$ und ungerade Punkte $x^{\prime\prime}_k$ ein, so folgt für die DFT:
		
		\begin{align}
		f_m &= \sum_{k=0}^{n-1}x^{\prime}_k\cdot\mathrm{e}^{\left(-\frac{2\pi i}{n}mk\right)} \quad + \quad \mathrm{e}^{\left(-\frac{\pi i}{n}m\right)}\sum_{k=0}^{n-1}x^{\prime\prime}_k\cdot\mathrm{e}^{\left(-\frac{2\pi i}{n}mk\right)} \\
		&= \begin{cases} 
		f^{\prime}_m     + \mathrm{e}^{\left(-\frac{\pi i}{n}m\right)}    f^{\prime\prime}_m     & m < n \\
		f^{\prime}_{m-n} - \mathrm{e}^{\left(-\frac{\pi i}{n}(m-n)\right)}f^{\prime\prime}_{m-n} & m \geq n 
		\end{cases}
		\end{align}
		
		Mit der Berechnung von $f^{\prime}_m$ und $f^{\prime\prime}_m$ ist sowohl $f_m$ als auch $f_{m+n}$ bestimmt. Der Rechenaufwand hat sich durch diese Zerlegung also praktisch halbiert. 
		
		Durch eine Rekursion lässt sich diese Eigenschaft ausnutzen.
		\begin{enumerate}
		    \item Das Feld mit den Eingangswerten wird einer Funktion als Parameter übergeben, die es in zwei halb so lange Felder (eins mit den Werten mit geradem und eins mit den Werten mit ungeradem Index) aufteilt.
			\item Diese beiden Felder werden nun an neue Instanzen dieser Funktion übergeben.
			\item Am Ende gibt jede Funktion die FFT des ihr als Parameter übergebenen Feldes zurück. Diese beiden FFTs werden nun, bevor eine Instanz der Funktion beendet wird, nach der oben abgebildeten Formel zu einer einzigen FFT kombiniert - und das Ergebnis an den Aufrufer zurückgegeben.
		\end{enumerate}
		Ist $n$ eine zweier-Potenz, so lässt sich der Aufwand effektiv auf $O(n\log n)$ reduzieren. Diese Einschränkung stellt meist kein Hindernis dar, da die Wahl von $n$ in der Praxis oft beliebig ist.
		
		In einer iterativen Variante lässt sich dieser Algorithmus auf einer GPU parallelisieren: In jedem Schritt werden die geraden und ungeraden Punkte in zwei Hälften geteilt. Man erhält ein sortiertes Array, in dem von jeweils zwei Elementen die DFT gebildet werden kann. Im nächsten Schritt werden diese Werte zur DFT von vier Werten kombiniert. Dies führt man so lange durch, bis man auf der obersten Ebene angelangt ist. 
		
		Die inverse Transformation unterscheidet sich nur durch Vorzeichen und eine Normierungskonstante.
		
		Dieses Vorgehen ähnelt stark einer Reduktion. Es ist daher nicht überraschend, dass auch hier eine CUDA Library als Teil des Toolkits zur Verfügung steht.
		
		\subsection{cuFFT}
		Bei der cuFFT Library handelt es sich um die state-of-the-art Implementierung einer GPU parallelisierten FFT. Sie wurde im Wesentlichen von der \textit{Fastest Fourier Transformation in the West} (FFTW) inspiriert, unterscheidet sich aber in einem wichtigen Punkt (siehe \ref{fftw}). Zur Benutzung muss der Header \textit{cufft.h} (bzw. \textit{cufftXt.h} für Xt Funktionalität) inkludiert und mit \textit{libcufft} gelinkt werden.
		
		cuFFT unterstützt FFTs in bis zu drei Dimensionen von Komplex nach Komplex (CUFFT{\_}C2C), Real nach Komplex (CUFFT{\_}R2C) und Komplex nach Real (CUFFT{\_}C2R). R und C bezeichnen dabei einfache Präzision. Für die doppelte verwendet man D und Z.
	
		Zunächst müssen Arrays eines speziellen cuFFT Datentyps (\li`cufftComplex`. \li`cufftReal`, \li`cufftDouble`,\\
		\li`cufftDoubleComplex`) erstellt und wie gewohnt in den \gls{global Memory} kopiert werden:
		
		\begin{lstlisting}[caption=cuFFT: komplexer Datentyp]
		int dim = {Nx, Ny, Nz};
		cufftComplex *data;
		cudaMalloc(&data, Nx*Ny*Nz * sizeof(cufftComplex));
		\end{lstlisting}
		
		Ein Wert lässt sich dann wie in einem multidimensionalen Array von zweikomponentigen Datenstrukturen setzen, z.B. \li`data[x][y][z].x = 5.0f; data[x][y][z].y = 3.0f;` für $5 + i\cdot 3$.
	
		Im nächsten Schritt muss ein \Gls{Handle} für das Programm, ein sogenannter Plan erstellt werden. Dies ist notwendig, da abhängig von der Beschaffenheit der Daten vorab bereits verschiedene Berechnungen durchgeführt werden müssen, z.B. die Größe der \Glspl{Block} oder welcher Algorithmus benutzt werden soll. 
		
		\begin{lstlisting}[caption=cuFFT: Pläne]
		cufftHandle plan;
		/////////in einer Dimension /////////
		cufftPlan1d(&plan, Nx, CUFFT_C2C, 1);	
		/////////in zwei Dimensionen/////////
		cufftPlan2d(&plan, Nx, Ny, CUFFT_C2C, 1);	
						
		/////////in drei Dimensionen/////////
		cufftPlan3d(&plan, Nx, Ny, Nz, CUFFT_C2C, 1);
		\end{lstlisting}
		
		Der letzte Wert bezeichnet die Batch Größe (später mehr).
	    
		Dieser Plan muss nun ausgeführt und evtl. das Ergebnis zur Ausgabe zurück kopiert werden.
		
		\begin{lstlisting}[caption=cuFFT: Ausführen]
		cufftExecC2C(plan, data, data, CUFFT_FORWARD);
		cufftDestroy(plan);
		cudaFree(data);
		\end{lstlisting}
		
		Input und Output Array sind hier identisch, es handelt sich also um eine in-place Transformation. Das letzte Keyword bezeichnet die Art der Transformation. CUFFT\_ INVERSE bezeichnet die inverse Transformation. Da Input und Output Array nicht zwingend vom selben Datentyp sein müssen, können ihre Größen abweichen. Tabelle \ref{tab6:size} zeigt einen Vergleich.
		
		\begin{table}[h]
		    \centering
		    \begin{tabular}{llll}
			    \toprule
        			Dimension & FFT & Inputgröße & Outputgröße \\ \midrule
        		    	1D & C2C & $N_x$ \li`cufftComplex` & $N_x$ \li`cufftComplex` \\
			    1D & C2R & $\left \lfloor{\frac{N_x}{2}}\right \rfloor + 1$ \li`cufftComplex` & $N_x$ \li`cufftReal` \\
		        	1D & R2C & $N_x$ \li`cufftReal` & $\left \lfloor{\frac{N_x}{2}}\right \rfloor + 1$ \li`cufftComplex` \\ \hline
        			2D & C2C & $N_x\cdot N_y$ \li`cufftComplex` & $N_x\cdot N_y$ \li`cufftComplex` \\
		        	2D & C2R & $N_x\left \lfloor{\frac{N_y}{2}}\right \rfloor + 1$ \li`cufftComplex` & $N_x\cdot N_y$ \li`cufftReal` \\
        			2D & R2C & $N_x\cdot N_y$ \li`cufftReal` & $N_x\left \lfloor{\frac{N_y}{2}}\right \rfloor + 1$ \li`cufftComplex` \\ \hline
		        	3D & C2C & $N_x\cdot N_y\cdot N_z$ \li`cufftComplex` & $N_x\cdot N_y\cdot N_z$ \li`cufftComplex` \\
        			3D & C2R & $N_x\cdot N_y\left \lfloor{\frac{N_z}{2}}\right \rfloor + 1$ \li`cufftComplex` & $N_x\cdot N_y\cdot N_z$ \li`cufftReal` \\
		        	3D & R2C & $N_x\cdot N_y\cdot N_z$ \li`cufftReal` & $N_x\cdot N_y\left \lfloor{\frac{N_z}{2}}\right \rfloor + 1$ \li`cufftComplex` \\ \bottomrule
        		\end{tabular}
		    \caption[cuFFT Arraygrößen]{Größenvergleich von Input und Output Arrays}
		    \label{tab6:size}
		\end{table}
	
		Nvidia stellt auch hier eine Dokumentation im gewohnten Format zur Verfügung. \autocite{cufftDoc}
	
		 	\subsubsection{cuFFT v.s. FFTW}\label{fftw}
		 	Die \textit{Fastest Fourier Transformation in the West} (FFTW) hat sich wegen ihrer freien Zugänglichkeit und ihrer effizienten Parallelisierung in OpenMP und MPI für Prozessoren als de-facto Standard durchgesetzt. Sie unterscheidet sich in einem wesentlichen Punkt gegenüber cuFFT: FFTW stellt viele Pläne zur Verfügung und führt sie auf eine Art aus, cuFFT stellt nur drei Pläne zur Verfügung und benutzt mehrere Methoden zur Ausführung \autocite{FFTW05}. Tabelle \ref{tab6:fftw} zeigt einige Unterschiede.
		 	\begin{table}[h]
		 	\centering
		 	\begin{tabular}{ll}
		 		\toprule
		 		FFTW & cuFFT \\ \midrule
		 		\li`fftw_plan_dft_1d()`, \li`fftw_plan_dft_r2c_1d()`, \li`fftw_plan_dft_c2r_1d()` & \li`cufftPlan1d()` \\
		 		\li`fftw_plan_dft_2d()`, \li`fftw_plan_dft_r2c_2d()`, \li`fftw_plan_dft_c2r_2d()` & \li`cufftPlan2d()` \\
		 		\li`fftw_plan_dft_3d()`, \li`fftw_plan_dft_r2c_3d()`, \li`fftw_plan_dft_c2r_3d()` & \li`cufftPlan3d()` \\
		 		\li`fftw_plan_dft()`, \li`fftw_plan_dft_r2c()`, \li`fftw_plan_dft_c2r()` & \li`cufftPlanMany()` \\
		 		\li`fftw_plan_many_dft()`, \li`fftw_plan_many_dft_r2c()`, \li`fftw_plan_many_dft_c2r()` & \li`cufftPlanMany()` \\
		 		\li`fftw_execute()` & \multirowcell{6}{\li`cufftExecC2C()`\\ \li`cufftExecZ2Z()`\\ \li`cufftExecR2C()`\\ \li`cufftExecD2Z()`\\ \li`cufftExecC2R()`\\ \li`cufftExecZ2D()`} \\
		 		& \\
		 		& \\
		 		& \\
		 		& \\
		 		& \\
		 		\li`fftw_destroy_plan()` & \li`cufftDestroy()` \\ \bottomrule
		 	\end{tabular}
		 	\caption[cuFFT v.s. FFTW]{Unterschiede zwischen FFTW und cuFFT}
		 	\label{tab6:fftw}
		 	\end{table}
		 	
		 	Eine eindimensionale Transformation würde mit FFTW in der OpenMP-parallelen Variante so aussehen:
		 	\begin{lstlisting}[caption=FFTW Beispiel] 
		 	fftw_complex *in = (fftw_complex*) fftw_malloc(sizeof(fftw_complex)*N);
		 	//Komplexen Vektor belegen...
			fftw_init_threads();
			fftw_plan_with_nthreads(omp_get_max_threads());	
			fftw_plan p = fftw_plan_dft_1d(N, in, in, FFTW_FORWARD, FFTW_ESTIMATE); 
			fftw_execute(p); 
			
			fftw_destroy_plan(p);
			fftw_free(in);
			fftw_cleanup_threads();	 	
		 	\end{lstlisting}
		 	
			Abbildung \ref{fig6:fft} zeigt einen Laufzeitvergleich.
		
			Mehr Informationen lassen sich in der Dokumentation nachlesen. \autocite{fftwDoc}
			
			\subsubsection{cuFFTW}
			Um eine Umstellung zu erleichtern, wurde ein \Gls{API} programmiert, das manche FFTW Funktionen auf cuFFT abbildet. Dazu muss lediglich der Header gegen \textit{cufftw.h} ausgetauscht und mit \textit{libcufftw} gelinkt werden. Kapitel 7 in der Dokumentation zeigt alle implementierten Funktionen. \autocite{cufftDoc}		
			
		\subsection{clFFT}
		Es existiert eine Open-Source Implementierung der FFT in OpenCL von AMD. Sie wurde zwar für AMD GPUs optimiert lässt sich aber prinzipiell auf allen benutzen. Diese steht unter \url{https://github.com/clMathLibraries/clFFT} zur Verfügung und muss explizit nach Anleitung kompiliert werden. Zur Benutzung muss der Header \textit{clfft.h} inkludiert und mit \textit{libclFFT} (und wie gewohnt mit \textit{libOpenCL}) gelinkt werden. Eine eindimensionale Transformation würde mit clFFT so aussehen:
		
		\begin{lstlisting}[caption=clFFT Beispiel]
		cl_int err, N = ...;
		float *X = (float*)malloc(N*2 * sizeof(*X));
		//Komplexen Vektor belegen...
		
		cl_context ctx = ...;
		cl_command_queue queue = ...;
		
		clfftSetupData fftSetup;
		clfftInitSetupData(&fftSetup);
		clfftSetup(&fftSetup);

		cl_mem bufX = clCreateBuffer(ctx, CL_MEM_READ_WRITE, 
			N*2 * sizeof(*X), NULL, &err);
		clEnqueueWriteBuffer(queue, bufX, CL_TRUE, 0, 
			N*2 * sizeof(*X), X, 0, NULL, NULL);
		
		clfftDim dim = CLFFT_1D;
		size_t clLengths[1] = {N};
		clfftPlanHandle planHandle;
		clfftCreateDefaultPlan(&planHandle, ctx, dim, clLengths);

		clfftSetPlanPrecision(planHandle, CLFFT_SINGLE);
		clfftSetLayout(planHandle, CLFFT_COMPLEX_INTERLEAVED, 
            CLFFT_COMPLEX_INTERLEAVED);
		clfftSetResultLocation(planHandle, CLFFT_INPLACE);

		clfftBakePlan(planHandle, 1, &queue, NULL, NULL);
		clfftEnqueueTransform(planHandle, CLFFT_FORWARD, 1, 
			&queue, 0, NULL, NULL, &bufX, NULL, NULL);
		
		.........
		
		clfftDestroyPlan(&planHandle);
		clfftTeardown();
		\end{lstlisting}		
		
		Abbildung \ref{fig6:fft} zeigt einen Laufzeitvergleich.	
					
		\begin{figure}[h]
  			\centering
			\begin{tikzpicture}
    			    \begin{loglogaxis}[xlabel={vector size $n$}, ylabel={computation time / msec.}, legend pos=outer north east, legend style={cells={align=left}}]
      		        \addplot [draw=UR@color@12!50!black, mark=*, only marks, mark options={scale=1}, fill=UR@color@12]   table[x index=0, y index=1]{chapter6/plots/fft.dat};
      		        \addlegendentry{OpenMP FFTW}
%
    	  		        \addplot [draw=gray!50!black, mark=*, only marks, mark options={scale=1}, fill=gray!50!white] table[x index=0, y index=2]{chapter6/plots/fft.dat};
    	  		        \addlegendentry{clFFT}
%
      		        \addplot [draw=UR@color@12!30!black, mark=*, only marks, mark options={scale=1}, fill=UR@color@12!30!white] table[x index=0, y index=3]{chapter6/plots/fft.dat};
      		        \addlegendentry{cuFFT}  
		   	    \end{loglogaxis}
            \end{tikzpicture}
  			\caption[Vergleich von FFTs]{Laufzeitvergleich von \textit{Complex-to-Complex} FFTs desselben $n$-dimensionalen Vektors (log-log). Zum Einsatz kam eine \textit{Nvidia GTX 1060} sowie ein Intel i7-8700K 6$\times$4.8GHz.}
  			\label{fig6:fft}
		\end{figure}
		
		Mehr Informationen zum \Gls{API} lassen sich in der Dokumentation nachlesen. \autocite{clfftDoc}
	\stopcontents[Fast Fourier Transformation]
		
	\newpage
	\startcontents[Lineare Algebra]
    \printcontents[Lineare Algebra]{l}{2}{\section*{Lineare Algebra}\setcounter{tocdepth}{2}}
		\section{Lineare Algebra}
		\subsection{cuBLAS und clBLAS}
		Um für Programmierer das Implementieren von Algorithmen in der linearen Algebra zu erleichtern, wurden in den 70er Jahren in Fortran verschiedenen Matrix- und Vektoroperationen hardwarenah implementiert. Diese Bibliothek ist als \textit{Basic Linear Algebra Subprograms} (BLAS) bekannt. Es exisiteren Wrapper und Portierungen für verschiedene Sprachen.
	
		BLAS besteht aus drei Teilen:
		\begin{itemize}
			\item \textbf{Level 1: Vektor-Vektor Operationen}\\
			Normen, Skalarprodukte, Dotprodukte, Kreuzprodukte, ...
			\item \textbf{Level 2: Matrix-Vektor Operationen}\\ Multiplikationen von Vektoren und Matrizen unter Nebenbedingungen (z.B. symmetrische Matrix, ...)
			\item \textbf{Level 3: Matrix-Matrix Operationen}\\Matrixnormen, Matrixadditionen, Matrixmultiplikationen unter Nebenbedingungen, ...
		\end{itemize}
		
		Es existiert eine Implementierung von Nvidia in CUDA namens cuBLAS. In dieser Bibliothek, die Teil des CUDA Toolkits ist, werden diese Operationen auf der GPU parallelisiert. Zur Benutzung muss der Header \textit{cublas{\_}v2.h} (in älteren Versionen \textit{cublas.h}) inkludiert und mit \textit{libcublas} gelinkt werden.
		
		Die Handhabung soll möglichst einfach gehalten werden. Der Anwender implementiert Vektoren als gewöhnliche Arrays und linearisierte Matrizen ebenfalls als Array. Für komplexe Zahlen existiert das Datenformat \li`cuComplex` (\li`cuDoubleComplex`), welches analog zu \li`cufftComplex` (\li`cufftDoubleComplex`) funktioniert. In cuBLAS geht man von column-major Format und 1-based indexing aus. Dabei wird eine Matrix spaltenweise nacheinander in einem Array abgelegt. Dann ist für ein Element einer Matrix $A(i,j)$ (wobei $i$ und $j$ mit 1 beginnen) der Index im Array \li`A[(j-1)*m + i-1]`, wobei $m$ die Anzahl an Reihen einer Matrix ist. 
		
		\begin{lstlisting}[caption=cuBLAS: Matrix setzen]
		#define IDX2F(i,j,ld) ((((j)-1)*(ld))+((i)-1))
		
		uint M = ...;
		uint N = ...;	

		float* a = (float *)malloc (M * N * sizeof(float));
		for (uint j = 1; j <= N; j++) 
		{
			for (uint i = 1; i <= M; i++) a[IDX2F(i,j,M)] = ...;
		}
		float *b = ...; float *c = ...;

		float* devPtrA;
		cudaMalloc(&devPtrA, M*N*sizeof(float));

		float* devPtrA;
		float* devPtrB;
		float* devPtrC;
		cublasSetMatrix(M, N, sizeof(float), a, M, devPtrA, M);
		...
		\end{lstlisting}
		
		Sobald die Objekte gesetzt wurden, wird ähnlich wie in cuFFT ein \Gls{Handle} erstellt. Dann können auf diesen Objekten bestimmte Operationen ausgeführt werden, deren Namen und Handhabung in Kapitel 2.4 der cuBLAS Dokumentation \autocite{cublasDoc} nachgeschlagen werden können. In den meisten Fällen existieren Funktionen in mehreren Versionen für verschiedenen Präzisionen. Nach Ausführung kann das Objekt zur Ausgabe in den CPU Speicher zurückkopiert werden. Typischerweise implementieren diese Funktionen eine Abfolge von mehreren Operationen, die auf eine spezielle Operation angepasst werden muss.
		
		Man betrachte die Berechnung einer Matrixmultiplikation $C = A\cdot B$ einer $m\times n$ Matrix $A$ und einer $n\times m$ Matrix $B$ im allgemeinen Fall für einfache Präzision. Dafür existiert die Funktion \li`cublasSgemm`. Diese implementiert die Abbildung $C \leftarrow \alpha\cdot\textbf{op}(A)\textbf{op}(B) + \beta\cdot C$, wobei op() für transponieren oder konjugieren stehen kann und $\alpha$ bzw. $\beta$ beliebige Skalare sind. Für op() existiert der Datentyp \li`cublasOperation_t` und kann entweder \li`CUBLAS_OP_N` (\enquote{normal}), \li`CUBLAS_OP_T` (\enquote{transponiert}) oder \li`CUBLAS_OP_C` (\enquote{komplex konjugiert}) sein. In diesem Beispiel muss also die Funktion wie folgt gerufen werden: 
		
		\begin{lstlisting}[caption=cuBLAS: Funktionsaufruf]
		cublasHandle_t handle;
		cublasCreate(&handle);
		
		cublasSgemm(handle, 
			CUBLAS_OP_N , CUBLAS_OP_N, M, M, N, 
			1.0, devPtrA, M, devPtrB, N, 
			0.0, devPtrC, M);
		
		cublasGetMatrix(M, M, sizeof(float), devPtrC, M, c, M);
		
		cudaFree(devPtrA); cudaFree(devPtrB); cudaFree(devPtrC);			
		free(a); free(b); free(c);
		
		cublasDestroy(handle);
		\end{lstlisting}
		
		Zunächst muss eine $m\times m$ Matrix $C$ mit null vorbelegt werden. Die Größen von $A$ und $B$ müssen richtig angegeben und mit der Operation \li`CUBLAS_OP_N` versehen werden. Dazu wird nach dem Devicepointer der Matrix die Leading Dimension angegeben. Die zweite Dimension ergibt sich aus der Gesamtgröße des Arrays. Zudem müssen Länge und Breite noch einmal explizit angegeben werden, um evtl. nur eine Submatrix zu multiplizieren. Skalar $\alpha$ wird auf eins, $\beta$ auf null gesetzt. 
		
		In älteren Versionen sind Wrapper für die Allozierungsfunktionen vorhanden. Diese gelten als veraltet und sollten nicht mehr verwendet werden.
		
		Es existiert eine von AMD entwickelte Portierung clBLAS für OpenCL. Wie gewohnt ist diese Bibliothek quelloffen und kann unter \url{https://github.com/clMathLibraries/clBLAS} heruntergeladen werden. Auch diese Library wurde für AMD GPUs optimiert, lässt sich aber auch auf anderen Geräten verwenden. Eine Dokumentation steht ebenfalls zur Verfügung \autocite{clblasDoc}. Nach Installation muss der Header \textit{clblas.h} inkludiert und mit \textit{libclBLAS} (und wie gewohnt mit \textit{libOpenCL}) gelinkt werden.

		Zur Benutzung muss eine Initialisierungsfunktion gerufen werden. Matrizen werden ebenfalls im column-major Format in einen Buffer geschrieben. Da sich clBLAS ebenfalls an BLAS orientiert, unterscheiden sich die Namen der Funktionen gegenüber cuBLAS nur im Präfix \enquote{cl}. Nach Ausführen dieser Funktion muss synchronisiert und das Ergebnis zurückkopiert werden.\\ \\
		\begin{lstlisting}[caption=clBLAS Beispiel]
		cl_int err;
		cl_uint M = ...; cl_uint N = ...;
		
		cl_float *A = ...; cl_float *B = ...; cl_float *C = ...;
		
		clblasSetup();
		
		cl_mem bufA = clCreateBuffer(ctx, CL_MEM_READ_ONLY,  
			M*N * sizeof(cl_float), NULL, &err); 
		cl_mem bufB = ...; cl_mem bufC = ...;

		clEnqueueWriteBuffer(queue, bufA, CL_TRUE, 0, 
			M*N * sizeof(cl_float), A, 0, NULL, NULL);       
		 
		cl_event event;
		clblasSgemm(clblasColumnMajor, clblasNoTrans, clblasNoTrans, M, M, N,
			1.0, bufA, 0, M, bufB, 0, N, 
			0.0, bufC, 0, M,
			1, &queue, 0, NULL, &event);
		clWaitForEvents(1, &event);

		clEnqueueReadBuffer(queue, bufC, CL_TRUE, 0, 
			M*M * sizeof(cl_float), C, 0, NULL, NULL);

		clReleaseMemObject(bufC);clReleaseMemObject(bufB);clReleaseMemObject(bufA);
		clblasTeardown();
		\end{lstlisting}
		
		Einziger Unterschied zu cuBLAS ist die Angabe von OpenCl-spezifischen Eigenheiten, wie z.B. die Angabe eines Offsets oder	
der Events.
		
		\subsection{cuSPARSE und clSPARSE}
		In vielen Fällen der Praxis handelt es sich bei Matrizen um sogenannte dünn besetzte Matrizen (engl. \textit{sparse matrices}). Das Prinzip eines sparse-Formates beruht darauf, die wesentliche Anzahl von Nullen in der Matrix nicht zu speichern. Man speichert lediglich die von null verschiedenen Werte und deren Indizes in der Matrix. Bei einer Bandmatrix z.B. steigt die Zahl dieser Elemente linear mit $n$, die Zahl der Nullen aber quadratisch. Daher ist in den meisten Fällen ein solches Format bereits wegen dem geringeren Speicherbedarf sinnvoll. 
		
		Eine Multiplikation mit Null ergibt immer Null und muss nicht explizit ausgeführt werden. Um die Performance zu steigern, müssen folglich Algorithmen verwendet werden, die dieses Format explizit unterstützen. Für CUDA existiert die Bibliothek cuSPARSE. Das Inkludieren und Linken erfolgt wie gewohnt (\textit{cusparse.h}, \textit{libcusparse}).
		
		Folgendes Beispiel zeigt die Erstellung einer Matrix im \textit{Coordinated Format} (COO). Wie bei cuBLAS existiert auch hier kein expliziter Datentyp. Für die Matrixwerte und die Indizes werden eigene Arrays erstellt und manuell belegt.	
		\begin{lstlisting}[caption=cuSPARSE: Matrix erstellen]
		int n   = ...;
		int nnz = ...;
		int *cooRowIndexHostPtr =    (int*)malloc(nnz*sizeof(int));
		int *cooColIndexHostPtr =    (int*)malloc(nnz*sizeof(int));
		double *cooValHostPtr   = (double*)malloc(nnz*sizeof(double));


		//Beispiel (COO Format): 
		cooRowIndexHostPtr[0]=0; cooColIndexHostPtr[0]=0; cooValHostPtr[0]=1.0;
		cooRowIndexHostPtr[1]=0; cooColIndexHostPtr[1]=2; cooValHostPtr[1]=2.0;
		...		
		\end{lstlisting}

		Bei diesem Format werden die Werte und deren Indizes in der Matrix nacheinander, ohne Ordnung und zeilenweise abgelegt. Es existieren komplexere Formate, die für bestimmte Algorithmen notwendig sind. Für die Operation aus dem letzten Abschnitt wird beispielsweise das \textit{Block Compressed Sparse Row Format} (BSR) Format benötigt. Die genauen Anforderungen der Formate können in der Dokumentation nachgelesen werden. \cite{cusparseDoc}
		Für Vektoren ist die Angabe eines zweiten Index natürlich überflüssig. Dichte Vektoren werden als gewöhnliche Arrays gespeichert.
		\begin{lstlisting}[caption=cuSPARSE: Vector erstellen]
		int nnz_vector = ;
		int    *xIndHostPtr =    (int*)malloc(nnz_vector*sizeof(int));
		double *xValHostPtr = (double*)malloc(nnz_vector*sizeof(double));
    
		double *yHostPtr    = (double*)malloc(2*n       *sizeof(double));
    
		double *zHostPtr    = (double*)malloc(2*(n+1)   *sizeof(double));
		\end{lstlisting}

		Sämtliche Arrays müssen wie üblich auf der Grafikkarte alloziert und vom Host kopiert werden. Danach wird ein \gls{Handle} und ein Matrix Deskriptor erstellt, der eine Datenstruktur für das Device aufbaut. Die Angabe des Matrixtyps und der Art der Indizierung ist dabei erforderlich. Für Ersteres existieren folgende Möglichkeiten:
		\begin{itemize}
			\item CUSPARSE{\_}MATRIX{\_}TYPE{\_}GENERAL
			\item CUSPARSE{\_}MATRIX{\_}TYPE{\_}SYMMETRIC
			\item CUSPARSE{\_}MATRIX{\_}TYPE{\_}HERMITIAN
			\item CUSPARSE{\_}MATRIX{\_}TYPE{\_}TRIANGULAR
		\end{itemize}

		Dadurch lassen sich beim Speichern redundante Informationen vermeiden.

		Die Library implementiert zudem einige Konvertierungsfunktionen, z.B. vom COO ins CSR Format. Wichtig ist die Angabe von \li`nnz`, die Anzahl der von null verschiedenen Elemente.
		\begin{lstlisting}[caption=cuSPARSE: Initialisierung und Konvertierung]
		//Malloc und Memcopy
		...

		cusparseHandle_t handle;
		cusparseCreate(&handle);

		cusparseMatDescr_t descr;
		cusparseCreateMatDescr(&descr);
		cusparseSetMatType(descr,CUSPARSE_MATRIX_TYPE_GENERAL);
		cusparseSetMatIndexBase(descr,CUSPARSE_INDEX_BASE_ZERO);
    
		int *csrRowPtr; 
		cudaMalloc(&csrRowPtr, (n+1)*sizeof(int));
		cusparseXcoo2csr(handle, cooRowIndex, nnz, n, 
		                 csrRowPtr, CUSPARSE_INDEX_BASE_ZERO);
		\end{lstlisting}

		Auch hier existieren drei Level an Operationen.
		\begin{itemize}
			\item \textbf{Level 1: sparseVektor - denseVektor Operationen}\\
			\item \textbf{Level 2: sparseMatrix - denseVektor Operationen}\\
			\item \textbf{Level 3: sparseMatrix - mehrere Vektoren gespeichert als denseMatrix}
		\end{itemize}
		
		Bei Letzterem werden die Vektoren nacheinander im selben Array abgespeichert. Im Folgenden wird jeweils ein Beispiel gezeigt. 
		\begin{lstlisting}[caption=cuSPARSE: Beispiele und Cleanup]
		//Level 1
		cusparseDsctr(handle, nnz_vector, xVal, xInd, 
		              &y[n], CUSPARSE_INDEX_BASE_ZERO);


		//Level 2
		cusparseDcsrmv(handle,CUSPARSE_OPERATION_NON_TRANSPOSE, n, n, nnz,
		               &number, descr, cooVal, csrRowPtr, cooColIndex, 
		               &y[0], &number, &y[n]);

		//Level 3
		double *z; cudaMalloc(&z,   2*(n+1)*sizeof(double));
		cudaMemset(z, 0, 2*(n+1)*sizeof(double));
		cusparseDcsrmm(handle, CUSPARSE_OPERATION_NON_TRANSPOSE, n, 2, n,
		               nnz, &number, descr, cooVal, csrRowPtr, cooColIndex, 
		               y, n, &number, z, n+1);
               
		cusparseDestroyMatDescr(descr);
		cusparseDestroy(handle);
		\end{lstlisting}

		\Gls{Handle} und Deskriptor müssen explizit zerstört werden. Wie gewohnt müssen Arrays auf Host und Device freigegeben werden.

		\subsection{cuSOLVER}
		Es exisitiert eine Reihe von Bibliotheken, die aufbauend auf BLAS die wichtigsten Algorithmen der linearen Algebra implementieren, z.B. QR-Zerlegung oder Eigenwertberechnung. Die bekannteste ist das \textit{Linear Algebra Package} (LAPACK), das in Fortran77 geschrieben wurde (umgeschrieben auf Fortran90) und für die meisten Programmiersprachen portiert wurde. Es existieren hierfür auch \Glspl{API}, z.B. Armadillo oder JLapack.
		Um für dicht und dünn besetzte Matrizen schnelle Methoden für Berechnungen in der linearen Algebra bereit zu stellen, existiert eine Implementierung der wichtigsten Algorithmen in der Bibliothek cuSOLVER, ein Teil des CUDA Toolkits. Linken und Inkludieren erfolgt wie gewohnt (\textit{cusolver.h}, \textit{libcusolver}). Diese Bibliothek orientiert sich an LAPACK und besteht aus drei Teilen:
		\begin{itemize}
		\item \textbf{cuSolverDN: Dense LAPACK} 
		
		Methoden zur Lösung eines regulären dicht besetzten Gleichungssystems $Ax = b$ (QR- und LU-Zerlegung mit Pivotisierung, Cholesky Zerlegung für symmetrisch/hermitesche Matrizen, Bunch-Kaufmann-Zerlegung für symmetrisch indefinite Matrizen, Singulärwertzerlegung).
		
		\item \textbf{cuSolverSP: Sparse LAPACK}
		
		Methoden zur Lösung eines dünn besetzten Gleichungssystems $Ax = b$, im überbestimmten Fall durch Regression $x = \text{argmin}||Az-b||$, im unterbestimmten Fall durch Lösen von $A^Tx=b$. Für symmetriche Matrizen muss eine volle Matrix angegeben werden (Ausnahme: Bei Cholesky-Zerlegung reicht die untere linke Dreiecksmatrix.) Zusätzlich existiert eine Eigenwertzerlegung auf Basis der shift-inverse-power Methode. 
		
		\item \textbf{cuSolverRF: Refactorization}
		
		Methoden zur schnellen Lösung einer Menge von Gleichungssystemen $A_i x_i = b_i$. Bleibt das sparsity-pattern\footnote{Position der von null verschiedenen Elemente} der Matrizen $A_i$, sowie das Neuordnen zur Pivotisierung und zum Auffüllen während der LU-Zerlegung gleich, ist dieser Teil der Bibliothek anwendbar. Zuerst wird für $i=1$ eine volle LU-Zerlegung durchgeführt. Für die folgenden kann ein LU-Refaktorisierungsalgorithmus verwendet werden.	
		\end{itemize}
		Jede Operation ist grundsätzlich für die Datentypen \li`float`, \li`double`, \li`cuComplex`, und \li`cuDoubleComplex` implementiert. Für die Funktionen gelten bestimmte Namenskonventionen. Für Dense:
		
		\li`cusolverDn<t><operation>` 

		Der Datentyp wird mit \li`<t>` angegeben, S (single), D (double), C (complex single), Z (complex double) und X (generic). 
		
		Für Sparse:
		
		\li`cusolverSp[Host]<t>[<matrix data format>]<operation>[<output matrix data format>]<based on>`
		
		Die Angabe \li`[Host]` steht für die CPU-Implementierung. \li`[<matrix data format>]` steht für die Angabe des sparse-matrix Formats (z.B. \li`csr`). \li`[<output matrix data format>]` ist entweder \li`m` oder \li`v` für Matrix oder Vektor. \li`<based on>` beschreibt den Algorithmus (z.B. \li`qr`).
		
		Im Folgenden soll die Lösung eines regulären linearen Gleichungssystems $A_{m\times m}x=B$ mittels LU-Zerlegung ermittelt werden.
		
		\begin{lstlisting}[caption=cuSOLVER: Buffer]
		const int m = ...; const int lda = m; const int ldb = m;
		float A[lda*m] = {...};
		float B[m] = {...};    

		cusolverDnHandle_t cusolverH; cusolverDnCreate(&cusolverH);
		cudaStream_t stream;
		cudaStreamCreateWithFlags(&stream, cudaStreamNonBlocking);
		cusolverDnSetStream(cusolverH, stream);

		float *d_A; cudaMalloc(&d_A, sizeof(float)*lda*m);
		float *d_B; cudaMalloc(&d_B, sizeof(float)*m);

		cudaMemcpy(d_A, A, lda*m*sizeof(float), cudaMemcpyHostToDevice);
		cudaMemcpy(d_B, B,     m*sizeof(float), cudaMemcpyHostToDevice);

		int lwork; float *d_work; 
		cusolverDnDgetrf_bufferSize(cusolverH, m, m, d_A, lda, &lwork);
		cudaMalloc(&d_work, sizeof(float)*lwork);
		\end{lstlisting}
		
		Für die Nutzung der Funktionen muss ein \Gls{Handle} erzeugt und einem \Gls{Stream} hinzugefügt werden. Belegen der Matrizen erfolgt wie gewohnt linearisiert. Allerdings muss danach mit \li`cusolverDnDgetrf_bufferSize` die Größe eines Buffers, der für die Zerlegung notwendig ist, ermittelt werden. Mit dieser Angabe wird dann der Buffer explizit angelegt.
		
		\begin{lstlisting}[caption=cuSOLVER: LU-Zerlegung]
		int Ipiv[m];      
		int info; 
		int *d_Ipiv; cudaMalloc(&d_Ipiv, m*sizeof(int));
		int *d_info; cudaMalloc(&d_info,   sizeof(int));
		
		const int pivot_on = 1;
		if(pivot_on) 
			cusolverDnDgetrf(cusolverH, m, m, d_A, lda, 
			d_work, d_Ipiv, d_info);
		else          
			cusolverDnDgetrf(cusolverH, m, m, d_A, lda, 
			d_work, NULL,   d_info);
			
		cudaDeviceSynchronize();

		float LU[lda*m];
		if(pivot_on) 
			cudaMemcpy(Ipiv, d_Ipiv, m*sizeof(int), cudaMemcpyDeviceToHost);
		cudaMemcpy(LU, d_A, lda*m*sizeof(float), cudaMemcpyDeviceToHost);
		cudaMemcpy(&info, d_info, sizeof(int), cudaMemcpyDeviceToHost);

		if(0 > info){	
			printf("%d-th parameter is wrong \n", -info); exit(1);
		}
		\end{lstlisting}
		
		Nun kann explizit eine LU-Zerlegung $A = LU$ mittels \li`cusolverDnDgetrf` durchgeführt werden. Dies funktioniert optional mit einer Pivotisierung $PA = LU$ (Permutationsmatrix $P$), bei der zusätzlich das Pivot Element jeder Zeile gespeichert wird. Bei dem speziellen Info \enquote{Array} \li`d_info` handelt es sich um einen einzigen globalen Integer-Wert, der mit negativem Vorzeichen angibt, welcher Parameter bei der Berechnung fehlerhaft war. Bei einer fehlerfreien Berechnung ist dieser Wert mit null belegt. Die Ausgabe der fertige Zerlegung wird als $(L-1)+U$ in einer einzigen Matrix gespeichert.
		
		\begin{lstlisting}[caption=cuSOLVER: Gleichungssystem lösen]
		if(pivot_on) 
			cusolverDnDgetrs(cusolverH, CUBLAS_OP_N, m, 1, d_A, lda, 
			d_Ipiv, d_B, ldb, d_info);
		else         
			cusolverDnDgetrs(cusolverH, CUBLAS_OP_N, m, 1, d_A, lda, 
			NULL,   d_B, ldb, d_info);
			
		cudaDeviceSynchronize();

		float X[m];
		cudaMemcpy(X, d_B, m*sizeof(float), cudaMemcpyDeviceToHost);

		cudaFree(d_A); cudaFree(d_B); cudaFree(d_Ipiv);
		cudaFree(d_info); cudaFree(d_work);

		cusolverDnDestroy(cusolverH);
		cudaStreamDestroy(stream);
		\end{lstlisting}
		
		Der Aufruf führt \li`cusolverDnDgetrs` (wahlweise mit Pivotisierung) eine Vor- und Rückwärtssubstitution aus und berechnet so die Lösung des Gleichungssystems ($Ly = B$, $Ux = y$). Das Ergebnis steht in \li`d_B` bereit.
		
		Für eine Operation aus Level 2 muss eine dünne Matrix im richtigen Format analog zu cuSPARSE erstellt werden. Dann wir ein entsprechender \Gls{Handle} mit \li`cusolverSpHandle_t` erstellt und die entsprechende Funktion ausgeführt. Da dies recht aufwändig ist und keinen großen Mehrwert bietet, soll hier darauf verzichtet werden.
		
		\subsection{MAGMA}
		Die zunehmende Notwendigkeit, Algorithmen auf heterogenen Systemen zu parallelisieren, motivierte die Entwicklung der \textit{Matrix Algebra on GPU and Multicore Architectures} (MAGMA). MAGMA orientiert sich in der Funktionsweise an BLAS bzw. LAPACK und soll effiziente Algorithmen implementieren, die die Hardware eines heterogenen Systems (gleichzeitig Vielkernprozessor und verschiedene GPUs) vollständig ausreizen. Bei Installation wird Existenz und Version von CUDA und OpenMP abgefragt. Dann wird die Bibliothek für alle relevanten Architekturen und Konfigurationen kompiliert. Linken erfolgt dann gegen \textit{libmagma} oder \textit{libmagma{\_}sparse} und gegen die entsprechenden Abhängigkeiten (\textit{liblapack}, \textit{libcuBLAS}, ...). Inkludiert wird \textit{magma{\_}v2.h} für die BLAS- und \textit{magma{\_}lapack.h} für die LAPACK-Portierung.

		Es existieren zwei Interfaces. Falls keine Grafikkarte zur Verfügung steht, kann das CPU-Interface benutzt werden. In diesem Fall wird ein OpenMP parallelisierter Algorithmus verwendet. Andernfalls kann mittels GPU-Interface ein mit CUDA beschleunigter Hybridalgorithmus verwendet werden. Sollte trotz vorhandener GPU (und GPU-Installation) das CPU-Interface benutzt werden, wird dennoch ein Hybridalgorithmus verwendet.
		
		Im Folgenden soll die LU-Zerlegung in cuSOLVER mit MAGMA durchgeführt werden.
		\begin{lstlisting}[caption=MAGMA: CPU-Interface]
		magma_init();
		magma_queue_t queue;
		magma_int_t  dev = 0;
		magma_queue_create(dev, &queue);
  
		magma_int_t m = ...; magma_int_t n = 1;
                      
		float *a; magma_smalloc_pinned(&a, m*m);   
		float *b; magma_smalloc_pinned(&b, m);  
		float *x; magma_smalloc_pinned(&x, m);   
		magma_int_t *piv = (magma_int_t *)malloc(m*sizeof(magma_int_t));
		
		//Belegen...

		const float alpha = 1.0f;
		const float beta  = 0.0f;
		blasf77_sgemm("N", "N", &m, &n, &n, &alpha, a, &m, x, &m, &beta, b, &m);

		magma_int_t info;
		magma_sgesv(m, n, a, m, piv, b, m, &info);

		magma_free_pinned(a); magma_free_pinned(b); magma_free_pinned(x); free(piv);

		magma_queue_destroy(queue);
		magma_finalize();
		\end{lstlisting}
		
		Zunächst wird MAGMA initialisiert. Danach wird ein Device ausgewählt und einer Command Queue hinzugefügt. Bei der Verwendung des CPU Interfaces für einen GPU Algorithmus wird managed Memory verwendet. Daher muss für die Arrays pinned Memory (page-locked) alloziert werden. Danach erfolgt ein Aufruf einer Matrixmultiplikation analog zu cuBLAS \mbox{(\li`blasf77_sgemm`)}, um die Lösung später überprüfen zu können. Die Zerlegung selbst erfolgt dann analog zu cuSOLVER mittels \li`magma_sgesv`.	
		
		\begin{lstlisting}[caption=MAGMA: GPU-Interface]
		magma_init();
		magma_queue_t queue;
		magma_int_t  dev = 0;
		magma_queue_create(dev, &queue);
  
		magma_int_t m = ...; magma_int_t n = 1;
                      
		float *a; magma_smalloc_cpu(&a, m*m);   
		float *b; magma_smalloc_cpu(&b, m);  
		float *x; magma_smalloc_cpu(&x, m);   
		magma_int_t *piv = (magma_int_t *)malloc(m*sizeof(magma_int_t));
		
		//Belegen...

		const float alpha = 1.0f;
		const float beta  = 0.0f;
		blasf77_sgemm("N", "N", &m, &n, &n, &alpha, a, &m, x, &m, &beta, b, &m);

		////////////////////////////////////////////
		
		float *d_a; magma_smalloc(&d_a, m*m);
		float *d_b; magma_smalloc(&d_b, m);		
		magma_ssetmatrix(m, m, a, m, d_a, m, queue);
		magma_ssetmatrix(m, n, b, m, d_b ,m, queue);
		
		////////////////////////////////////////////

		magma_int_t info;
		magma_sgesv_gpu(m, n, d_a, m, piv, d_b, m, &info);
		
		////////////////////////////////////////////
		
		magma_sgetmatrix(m, n, d_b, m, x, m, queue);
		magma_free(d_a); magma_free(d_b);
		
		////////////////////////////////////////////
		
		free(a); free(b); free(x); free(piv);
		
		magma_queue_destroy(queue);
		magma_finalize();
		\end{lstlisting}
		
		Das GPU Interface funktioniert an sich ähnlich, es müssen lediglich Arrays explizit vom Host auf das Device kopiert werden. Daher kann nun der CPU Speicher gewöhnlich mit \li`magma_smalloc_cpu` $\!\!$alloziert werden. GPU Speicher wird mit \li`magma_smalloc` erzeugt und mit \li`magma_ssetmatrix` bzw. \li`magma_sgetmatrix` kopiert. Der Aufruf der LU-Zerlegung \li`magma_sgesv_gpu` ist beinahe identisch. CPU Speicher wird normal mit \li`free`, GPU Speicher mit \li`magma_free` freigegeben.	
		
		MAGMA implementiert ebenfalls Algorithmen für dünne Matrizen. Aus bekannten Gründen wird hier jedoch nicht weiter darauf eingegangen.
		
		\subsection{Iterative Verfahren}
		Bei den Verfahren aus BLAS und LAPACK handelt es sich um direkte Lösungsverfahren. Diese liefern im Erfolgsfall (bis auf numerische Ungenauigkeiten) die analytische Lösung. Bei den sogenannten iterativen Verfahren handelt es sich um numerische Lösungen ähnlicher Probleme, deren Genauigkeit aber durch die Abbruchbedingung der Iteration gesteuert werden kann. Man opfert also Genauigkeit zu Gunsten von Performance. 
		
			\subsubsection{Paralution}
			Paralution ist eine sehr umfangreiche C++-Bibliothek für indirekte (sparse) Lösungsverfahren von Differentialgleichungen oder Problemen der linearen Algebra. Im folgenden Beispiel soll die zweidimensionale stationäre Wärmeleitungsgleichung $\Laplace x = b$ auf einem diskretisierten $100\times 100$ Gitter gelöst werden. $x$ bezeichnet die Temperatur, $b$ eine Inhomogenität.
			\begin{lstlisting}[caption=Paralution Beispiel]		
			paralution::init_paralution();
			paralution::info_paralution();

			paralution::LocalVector<float> x;
			paralution::LocalVector<float> rhs;

			paralution::LocalStencil<float> stencil(Laplace2D);
			stencil.SetGrid(100);

			x.Allocate("x", stencil.get_nrow()); x.Zeros();
			rhs.Allocate("rhs", stencil.get_nrow()); rhs.Ones();

			paralution::CG<LocalStencil<float>, LocalVector<float>, double> ls;
			ls.SetOperator(stencil);
			ls.Build();
			stencil.info();

			ls.Solve(rhs, &x);

			paralution::stop_paralution();
			\end{lstlisting}
			
			Die Daten werden hier mit Konstruktoraufrufen von C++-Vektoren erzeugt. Deren Inhalte werden durch explizite Aufrufe von Member Funktionen gesetzt. Dann wird ein sogenannter Stencil als numerische Näherung für den 2d-Laplace Operator erzeugt. Das Gitter, auf welches dieser Stencil angewendet werden soll, wird auf 100 ($100\times 100$) gesetzt. Damit wird ein linear Solver (ls) erzeugt (\li`ls.Build()`). Mit diesem kann die Differentialgleichung nun gelöst werden (\li`ls.Solve(rhs, &x)`). An Stelle eines Laplace Operators lässt sich genauso eine Matrix verwenden, um beispielsweise eine Zerlegung auszuführen. Mit der Initialisierunsfunktion \li`init_paralution` können mehrere Parameter, wie z.B. die Abbruchbedingung der Iteration eingestellt werden. Dies kann im äußerst ausführlichen Handbuch nachgeschlagen werden. \autocite{para}
			 
			\subsubsection{AMGX}	
			Bei AmgX handelt es sich um eine ähnliche Bibliothek. Die Stärken liegen hierbei u.A. bei Krylov-Subspace Methoden. AmgX ist ein offizielles Partnerprojekt von Nvidia und kann unter \url{https://github.com/NVIDIA/AMGX} heruntergeladen werden. Zusätzliche Python Bindings sind unter \url{https://github.com/shwina/pyamgx} verfügbar. Das Handbuch enthält mehr Informationen über die Nutzung. \autocite{amgx}	
			
		\subsection{clSPARSE}
		Ein Teil von cuSPARSE und cuSOLVER sowie zusätliche iterative Verfahren wurden von AMD in OpenCL als clSPARSE implementiert. Das \Gls{API} wurde vollständig in C++ programmiert und ist kompatibel zum OpenCL C++ Wrapper. Die Bibliothek enthält folgende Funktionen:
		
		\begin{itemize}				
			\item Sparse Matrix - dense Vector multiply (SpM-dV)
			\item Sparse Matrix - dense Matrix multiply (SpM-dM)
			\item Sparse Matrix - Sparse Matrix multiply Sparse Matrix Multiply(SpGEMM) - Single Precision
			\item Iterative conjugate gradient solver (CG)
			\item Iterative biconjugate gradient stabilized solver (BiCGStab)
			\item Dense to CSR conversions (\& converse)
			\item COO to CSR conversions (\& converse)
			\item Functions to read matrix market files in COO or CSR format
    	\end{itemize}
    	
    	Folgendes Beispiel implementiert einen iterativen CG-Solver. Dazu werden eine dünne Matrix $A$ sowie zwei dichte Vektoren $x$ und $b$ als C++-Objekte erstellt. Es folgt das Setup und der \Gls{Handle}, hier \textit{Control} genannt. 
    	\begin{lstlisting}[caption=clSPARSE: Initialisieren]
		cldenseVector x;
		clsparseInitVector(&x);
		
		cldenseVector b;
		clsparseInitVector(&b); 

		clsparseCsrMatrix A;
		clsparseInitCsrMatrix(&A); 
    	
		clsparseSetup();
		clsparseCreateResult createResult = clsparseCreateControl(queue());
    	\end{lstlisting}

		Die Daten der C++ Klasse werden explizit gesetzt. Für die Grafikkarte muss dennoch Speicher alloziert und kopiert werden. Man beachte den eigenen Datentyp \li`clsparseIdx_t`. Das Format der Matrix (CSR) muss dem \Gls{Handle} bekannt gemacht werden (\enquote{Meta}).\\
    	\begin{lstlisting}[caption=clSPARSE: Vektoren und Matrizen setzen]
		A.num_nonzeros = ...;
		A.num_rows = ...;
		A.num_cols = ...;    
		A.values =      cl::clCreateBuffer(context(), CL_MEM_READ_ONLY, 
		                                   A.num_nonzeros*sizeof(float));

		A.col_indices = cl::clCreateBuffer(context(), CL_MEM_READ_ONLY, 
		                                   A.num_nonzeros*sizeof(clsparseIdx_t));

		A.row_pointer = cl::clCreateBuffer(context(), CL_MEM_READ_ONLY, 
		                                  (A.num_rows + 1)*sizeof(clsparseIdx_t)); 
		
		//Belegen und Kopieren...
                                     
		clsparseCsrMetaCreate(&A, createResult.control);
      
		x.num_values = A.num_cols;
		x.values = clCreateBuffer(context(), CL_MEM_READ_ONLY, 
		                          x.num_values*sizeof(float));
		clEnqueueFillBuffer(queue(), x.values, &number, sizeof(float), 0, 
		                    x.num_values*sizeof(float), 0, nullptr, nullptr);
                                    
		b.num_values = A.num_rows;
		b.values = clCreateBuffer(context(), CL_MEM_READ_WRITE, 
		                          b.num_values*sizeof(float));
		clEnqueueFillBuffer(queue(), b.values, &number, sizeof(float), 0, 
		                          b.num_values*sizeof(float), 0, nullptr, nullptr);
		\end{lstlisting}  
		
		Nach dem Setup kann ein eigener \Gls{Handle} für den Solver erstellt werden. Es wird ein diagonaler Vorkonditionierer verwendet. Die Konverkenzkriterien sind $\varepsilon_{rel} = 10^{-2}$ und $\varepsilon_{abs} = 10^{-5}$. Das Iterationslimit wird auf $1000$ gesetzt. Schließlich wird die eigentliche Funktion gerufen. Beide \Glspl{Handle} werden nach Gebrauch zerstört. Trotz C++ \Gls{API} müssen die Speicherobjekte freigegeben werden. Die Metainformation muss ebenfalls gelöscht werden.
		
		\newpage
		                    
		\begin{lstlisting}[caption=clSPARSE: Ausführen]
		clsparseCreateSolverResult solverResult = 
			clsparseCreateSolverControl(DIAGONAL, 1000, 1e-2, 1e-5);
		
		clsparseScsrcg(&x, &A, &b, solverResult.control, createResult.control);
		
		clsparseReleaseSolverControl(solverResult.control);
    
		clsparseReleaseControl(createResult.control);
		
		clsparseTeardown();  
		  
		clsparseCsrMetaDelete(&A);
		
		clReleaseMemObject(A.values);
		clReleaseMemObject(A.col_indices);
		clReleaseMemObject(A.row_pointer);

		clReleaseMemObject(x.values);
		clReleaseMemObject(b.values);
    	\end{lstlisting}
    	
    	Die Dokumentation ist unter \href{http://clmathlibraries.github.io/clSPARSE/} zu finden. Für das Error Handling muss neben dem üblichen \textit{clSPARSE.h} zusätzlich \textit{clSPARSE-error.h} inkludiert werden. Linken erfolgt gegen \textit{libclSPARSE}.
			
	\stopcontents[Lineare Algebra]

	\newpage
	\startcontents[Machine Learning]
    \printcontents[Machine Learning]{l}{2}{\section*{Machine Learning}\setcounter{tocdepth}{2}}	
		\section{Machine Learning}
	Aus Wikipedia \autocite{wikiML}:
	
	Maschinelles Lernen ist ein Oberbegriff für die „künstliche“ Generierung von Wissen aus Erfahrung: Ein künstliches System lernt aus Beispielen und kann diese nach Beendigung der Lernphase verallgemeinern. Dazu bauen Algorithmen beim maschinellen Lernen ein statistisches Modell auf, das auf Trainingsdaten beruht. Das heißt, es werden nicht einfach die Beispiele auswendig gelernt, sondern es „erkennt“ Muster und Gesetzmäßigkeiten in den Lerndaten. So kann das System auch unbekannte Daten beurteilen [...].
	
	Hauptanwendungsgebiete sind u.A. Klassifizierung, Filterung, Bild- und Spracherkennung, Regression oder Datenanalyse und Datenerzeugung.
	Man kann diese Algorithmen in drei Kategorien einteilen:
	
	\begin{itemize}
		\item \textbf{supervised Learning:} Eingabedaten durchlaufen ein Modell. Die Ausgabedaten werden mit einer Vorgabe verglichen. Aus der Differenz zwischen Ist- und Soll-Zustand kann ein Verlust ausgerechnet werden. Durch statistische Minimierung dieses Verlustes werden die Parameter des Modells angepasst. Falls es sich bei dieser Minimierung um eine auf Stochastik beruhender Methode handelt, spricht man von stochastischem Lernen.
		\newpage
		\item \textbf{reinforcement Learning:} Kann auch als supervised Learning angesehen werden. Statt einer Klassifikation der Ergebnisse in "richtig" und "falsch" wird dem Algorithmus hier eine Belohnung vorgegeben. Diese Methode synergiert am besten mit dem Lernverhalten des Menschen.
		\item \textbf{unsupervised Learning:} Der Algorithmus versucht selbständig zu den Eingangsdaten basierend auf den Charakteristika der Signale ein Modell zu erstellen. Die Anwendungsmöglichkeiten sind hier natürlich vergleichsweise begrenzt. Ein bekanntes Beispiel ist der EM-Algorithmus.
	\end{itemize}
	
	Zu den klassischen Algorithmen zählen z.B. Support Vector Machines (SVM), Clustering, Random Forest, Empirical Mode Decomposition (EMD), Principal Component Analysis (PCA), Independent Component Analysis (ICA), Single Spectrum Analysis (SSA), Informationstheorie und viele mehr. Diese beruhen im Wesentlichen auf einfacher Statistik in Kombination mit Methoden der linearen Algebra.  SVMs beispielsweise berechnen Trennebenen zwischen linear separierbaren Punkten im $n$-dimensinalem Raum. Bei einem Random Forest handelt es sich im Grunde um eine Mittelung über verschiedene Entscheidungsbäume.
	
	Allerdings erfreuen sich in modernen Anwendungen sogenannte neuronale Netze immer größerer Beliebtheit. Der folgende Abschnitt konzentriert sich lediglich auf supervised Learning mit neuronalen Netzen \textit{off-line}, also Lernen, bei dem die Daten des trainierten Modells zu jedem Zeitpunkt erhalten bleiben.
	
		\subsection{neuronale Netze}
		Die Idee, ein menschliches Gehirn mit Computern nachzubauen, ist nicht neu und vermutlich so alt wie die Idee eines Computers. Die Anfänge lassen sich bereits auf die 40er Jahre zurückführen. Die ersten Theorien wurden in den 60ern entwickelt. Allerdings wurde die Entwicklung erst in den letzten Jahren von Erfolg gekrönt. Dies hat im Wesentlichen zwei Gründe. Der erste Grund ist \textit{Big Data}: Durch den Aufstieg des Internets wurden internationale Konzerne wie Facebook oder Google zu Sammelbecken von Daten ungeahnten Ausmaßes. Wie bereits erwähnt ist eine gute Datengrundlage wichtig für das Training eines Modells. Diese Konzerne sind natürlich aus kommerziellen Gründen daran interessiert, Forschung in diesem Bereich zu finanzieren.
		
		Der andere Grund beruht auf der Technik. Bei einem neuronalen Netz handelt es sich um ein SIMD Problem. Durch das Stagnieren der \Gls{Performance} von CPUs (siehe \ref{2:hard}) und der Aufwändigkeit und Kosten von Rechenclustern, die für SIMD Probleme notwendig wären, sind CPUs für das Trainieren von Netzen kaum geeignet. Der Durchbruch von General-Purpose GPUs, sowohl auf hardware- als auch auch auf software-technischer Ebene, erlaubt es allerdings zu einem erschwinglichen Preis neuronale Netze auch bei kleinen Projekten mit kleinem Budget einzusetzen.
		
		Neuronale Netze bestehen typischerweise aus einer Ausgabeschicht ("Output Layer") und mehreren verborgenen Schichten ("Hidden Layer"), die die Eingangssignale mit der Ausgabeschicht verbinden. Diese Schichten bestehen aus einer Vielzahl von Neuronen, die Eingangassignale mit verschiedenen Gewichten verrechnen. Eine logistische Funktion als Aktivierung entscheidet, ob der so berechnete Output an die nächste Schicht weitergegeben wird. Im Falle einer einzigen Schicht spricht man von einschichtigen Netzen (Single-Layer). 
		
		Man unterscheidet verschiedene Typen:
		
		\begin{itemize}
			\item \textbf{Single-Layer Perzeptron:} Ein einzelnes Neuron.
			\item \textbf{Multi-Layer Perzeptron:} Mehrere Perzeptronen hintereinander.
			\item \textbf{Fully-Connected Layer:} Jedes Neuron dieser Schicht gibt seinen Output an jedes Neuron der folgenden Schicht weiter.
			\item \textbf{Rekurrente Schicht:} Neuronen geben den Output auch an sich selbst weiter.
			\item \textbf{Convolutional Neural Network (CNN):} Vor der Klassifizierung durchlaufen die Daten eine Faltung, also Neuronen, die nur einen Teil des Inputs erhalten und auch nur einen Teil weitergeben.
			\item \textbf{Deep-Belief Network:} Eine Kombination von Boltzmann Maschinen.
			\item \textbf{Boltzmann Maschine:} Ein Stochastisches rekurrentes neuronales Netz zum Erlernen einer Wahrscheinlichkeitsverteilung.
			
			...
		\end{itemize}
		
		Wichtigste moderne Anwendung ist wohl das CNN. Da Fully-Connected Netzwerke oft zu aufwändig sind, z.B. bei der Verarbeitung von 2d Daten wie Bildern, wird zunächst nur ein Teil des Inputs an ein Neuron weitergegeben. Dies kann z.B. ein quadratischer Ausschnitt aus einem Bild sein. Nach sogenanntem Pooling, dem Zusammenfassen mehrerer Neuronen (typischerweise $2\times 2$) zu einer Aktivierung, wird der entstandene Output mit Fully-Connected Layern klassifiziert. Im mathematischen Sinne handelt es sich bei diesem Vorgang um eine Faltung. Dies motiviert den Begriff "convolutional", also "faltend".
		
		Man betrachte ein Fully-Connected Network mit zwei Schichten aus drei und zwei Neuronen. Sei ein Input Signal $\vec{x} = (x_1, x_2, x_3)^T$ gegeben. Dann wird am ersten Neuron jeder dieser Inputs mit einem Gewicht multipliziert und die Summe $\Sigma = w^1_1x_1 + w^1_2x_2 + w^1_3x_3$ an eine Aktivierungsfunktion weitergeleitet. Für das zweite und dritte Neuron findet dies genauso statt. Dieser Vorgang entspricht einer Matrixmultiplikation
		\begin{equation}
		\Sigma = (\Sigma_1, \Sigma_2, \Sigma_3)^T = W\vec{x}		
		\end{equation}
		wobei die Einträge von $W$ die Gewichte $W_{ij} = w^i_j$ sind. Nach dieser Multiplikation mit einer $3\times 3$ Matrix werden die Output Signale mit der Aktivierung verrechnet und an die nächste Schicht weitergegeben (feed-forward). Da es sich bei dieser nur um zwei Neuronen handelt, wird das dreikomponentige Signal mit einer $2\times 3$ Multiplikation auf ein zweikomponentiges abgebildet. Dieses Signal wird nun mit dem Sollzustand verglichen. Aus der Differenz wird ein Verlust berechnet (mit einer Loss-Function). Auf Basis dessen werden die Gewichte mit einem Minimierungsverfahren rückwärts angepasst, um diesen Zustand besser abzubilden (Backpropagation). Dieser Vorgang wird iterativ wiederholt (Epochen), bis eine bestimmte Akkuratesse erreicht ist (Verhältnis richtiger Klassifikationen zur Gesamtzahl). Inwiefern diese Genauigkeit auf dem Trainingsset ein Indikator für das tatsächliche Verhalten ist, hängt von der Korrelation der Daten ab und muss daher mit einem Testset verifiziert werden. Test- und Trainingsset sollten ein repräsentativer Teil der Ausgangsdaten sein und keine Überschneidung aufweisen. Wenn das Modell sich zu gut auf die Trainingsdaten angepasst hat und eine deutlich schlechtere Genauigkeit auf den Testdaten liefert, so spricht man von Overfitting. CNNs gelten als besonders resistent gegen diesen Fehler.
		
		Verbindet man im genannten Beispiel nur einen Input mit einem Neuron, so ergibt sich eine mathematische Faltung und die Gewichtsmatrix $W$ reduziert sich auf $\text{diag}(w^1_1, w^2_2, w^3_3)$. Bei dieser Matrix handelt es sich um eine sparse Matrix (und in diesem Spezialfall sogar um eine Band- und Diagonalmatrix).
		
		\subsection{cuDNN}
		\autocite{cudnnDoc}Bei cuDNN handelt es sich um die von Nvidia zur Verfügung gestellte GPU beschleunigte Implementierung von DNN Primitives. Dies sind die folgenden Routinen:
		\begin{itemize}
    		\item Convolution (Forward und Backward), cross-Korrelation
    		\item Pooling (Forward und Backward)
    		\item Softmax (Forward und Backward)
    		\item Neuron Aktivierungen (Forward und Backward):
			\begin{itemize}			        
        		\item Rectified linear (ReLU)
        		\item Sigmoid
        		\item Tangens Hyperbolicus (TANH)
        	\end{itemize}
    		\item Tensor Transformationsfunktionen
    		\item Local Response- und batch-Normalisierung, Locally-Connected Network (Forward und Backward)
    	\end{itemize}
    	
    	Das folgende Beispiel zeigt eine einzelne Epoche eines CNN.
    	\begin{lstlisting}[caption=cuDNN: Tensor-Descriptors]    	
		cv::Mat image = ...;
    	
		cudnnHandle_t cudnn;
		cudnnCreate(&cudnn);

		cudnnTensorDescriptor_t input_descriptor;
		cudnnCreateTensorDescriptor(&input_descriptor);
		cudnnSetTensor4dDescriptor(input_descriptor, CUDNN_TENSOR_NHWC, 
			CUDNN_DATA_FLOAT, 1, 3, image.rows, image.cols);

		cudnnTensorDescriptor_t output_descriptor;
		cudnnCreateTensorDescriptor(&output_descriptor));
		cudnnSetTensor4dDescriptor(output_descriptor,
			CUDNN_TENSOR_NHWC, CUDNN_DATA_FLOAT, 1, 3, image.rows, image.cols);

		cudnnFilterDescriptor_t kernel_descriptor;
		cudnnCreateFilterDescriptor(&kernel_descriptor);
		cudnnSetFilter4dDescriptor(kernel_descriptor,
			CUDNN_DATA_FLOAT, CUDNN_TENSOR_NCHW, 3, 3, 3, 3);

		cudnnConvolutionDescriptor_t convolution_descriptor;
		cudnnCreateConvolutionDescriptor(&convolution_descriptor);
		cudnnSetConvolution2dDescriptor(convolution_descriptor, 1, 1, 1, 1,
			1, 1, CUDNN_CROSS_CORRELATION, CUDNN_DATA_FLOAT);
		\end{lstlisting}

		\begin{lstlisting}[caption=cuDNN: Konvolutions-Algorithmus]
		cudnnConvolutionFwdAlgo_t convolution_algorithm;
		cudnnGetConvolutionForwardAlgorithm(cudnn, input_descriptor, 
			kernel_descriptor, convolution_descriptor, 
			output_descriptor, CUDNN_CONVOLUTION_FWD_PREFER_FASTEST, 0, 
			&convolution_algorithm);

		size_t workspace_bytes = 0;
		cudnnGetConvolutionForwardWorkspaceSize(cudnn, input_descriptor, 
			kernel_descriptor, convolution_descriptor,
			output_descriptor, convolution_algorithm, &workspace_bytes);

		void* d_workspace{nullptr};
		cudaMalloc(&d_workspace, workspace_bytes);
		\end{lstlisting}

		\begin{lstlisting}[caption=cuDNN: Kernel]
		int image_bytes = 1 * 3 * image.rows * image.cols * sizeof(float);
		float* d_input{nullptr};
		cudaMalloc(&d_input, image_bytes);
		cudaMemcpy(d_input, image.ptr<float>(0), 
			image_bytes, cudaMemcpyHostToDevice);

		float* d_output{nullptr};
		cudaMalloc(&d_output, image_bytes);
		cudaMemset(d_output, 0, image_bytes);

		const float kernel_template[3][3] = {
			{1,  1, 1},
			{1, -8, 1},
			{1,  1, 1}
		};

		float h_kernel[3][3][3][3];
		for(int kernel = 0; kernel < 3; ++kernel)
		{
			for (int channel = 0; channel < 3; ++channel)
			{
				for (int row = 0; row < 3; ++row) 
				{
					for (int column = 0; column < 3; ++column)
					{
					h_kernel[kernel][channel][row][column] 
						= kernel_template[row][column];
					}
				}
			}
		}

		float* d_kernel{nullptr};
		cudaMalloc(&d_kernel, sizeof(h_kernel));
		cudaMemcpy(d_kernel, h_kernel, sizeof(h_kernel), cudaMemcpyHostToDevice);
    	\end{lstlisting}
    	
    	\begin{lstlisting}[caption=cuDNN: feed forward]
		const float alpha = 1, beta = 0;
		cudnnConvolutionForward(cudnn, &alpha, input_descriptor, d_input,
			kernel_descriptor,d_kernel, convolution_descriptor,convolution_algorithm,
			d_workspace, workspace_bytes, &beta,output_descriptor, d_output);

		float* h_output = new float[image_bytes];
		cudaMemcpy(h_output, d_output, image_bytes, cudaMemcpyDeviceToHost);

		...

		delete[] h_output;
		cudaFree(d_kernel);
		cudaFree(d_input);
		cudaFree(d_output);
		cudaFree(d_workspace);

		cudnnDestroyTensorDescriptor(input_descriptor);
		cudnnDestroyTensorDescriptor(output_descriptor);
		cudnnDestroyFilterDescriptor(kernel_descriptor);
		cudnnDestroyConvolutionDescriptor(convolution_descriptor);

		cudnnDestroy(cudnn);
    	\end{lstlisting}

		\subsection{Exkurs: Python Tensorflow}
		\autocite{tfDoc}
		
		\url{https://www.tensorflow.org/overview/}
		
		\autocite{kerasDoc}
		
		\begin{lstlisting}[caption=Keras/Tensorflow Beispiel]
		import tensorflow as tf
		mnist = tf.keras.datasets.mnist

		(x_train, y_train),(x_test, y_test) = mnist.load_data()
		x_train, x_test = x_train / 255.0, x_test / 255.0

		model = tf.keras.models.Sequential([
			tf.keras.layers.Flatten(input_shape=(28, 28)),
			tf.keras.layers.Dense(512, activation=tf.nn.relu),
			tf.keras.layers.Dropout(0.2),
			tf.keras.layers.Dense(10, activation=tf.nn.softmax)
		])
		model.compile(optimizer='adam', loss='sparse_categorical_crossentropy', 
			metrics=['accuracy'])

		model.fit(x_train, y_train, epochs=5)
		model.evaluate(x_test, y_test)
		\end{lstlisting}	
		
		\subsection{TensorRT}
		\autocite{trtDoc}
		
		\autocite{trtSmpl}
		
		\autocite{tftrtDoc}
		
		\begin{lstlisting}[caption=Tensorflow Modell einfrieren]
		from keras.models import load_model
		import keras.backend as K
		from tensorflow.python.framework import graph_io
		from tensorflow.python.tools import freeze_graph
		from tensorflow.core.protobuf import saver_pb2
		from tenrflow.python.training import saver as saver_lib

		def convert_keras_to_pb(keras_model, out_names, models_dir, model_filename):
			model = load_model(keras_model)
			K.set_learning_phase(0)
			sess = K.get_session()
			saver = saver_lib.Saver(write_version=saver_pb2.SaverDef.V2)
			
			checkpoint_path = saver.save(sess, 'saved_ckpt', 
				global_step=0, latest_filename='checkpoint_state')
				
			graph_io.write_graph(sess.graph, '.', 'tmp.pb')
			
			freeze_graph.freeze_graph('./tmp.pb', '',
				False, checkpoint_path, out_names,
				"save/restore_all", "save/Const:0",
				models_dir+model_filename, False, "")
		\end{lstlisting}

	\stopcontents[Machine Learning]    
		
	\newpage
	\startcontents[Cluster Computing mit NCCL]
    \printcontents[Cluster Computing mit NCCL]{l}{2}{\section*{Cluster Computing mit NCCL}\setcounter{tocdepth}{2}}	
		\section{Cluster Computing mit NCCL}
		\subsection{Kollektive Operationen}
		Die Notwendigkeit, Algorithmen für GPU-Cluster zu parallelisieren, motivierte die Implementierung der \textit{Nvidia Collective Communications Library} (NCCL). Diese enthält mehrere Operationen, die auf Daten angewandt werden, die auf bis zu vier GPUs verteilt sind. Dazu wird zunächst ein spezieller Kommunikator definiert. Dann wird das entsprechende Gerät mit der jeweiligen Nummer ausgewählt. Für jedes Gerät wird ein Buffer der selben Größe erstellt sowie ein eigener Stream erstellt.
	
		\begin{lstlisting}[caption=NCCL: Buffer und Streams]
		ncclComm_t comms[4];
		int nDev = ...; int devs[nDev] = { 0, ... };

		float** sendbuff = (float**)malloc(nDev*sizeof(float*));
		float** recvbuff = (float**)malloc(nDev*sizeof(float*));
		cudaStream_t* s = (cudaStream_t*)malloc(sizeof(cudaStream_t)*nDev);
		for (int i = 0; i < nDev; ++i) {
			cudaSetDevice(i);
			cudaMalloc(sendbuff + i, size * sizeof(float));
			cudaMalloc(recvbuff + i, size * sizeof(float));
			cudaStreamCreate(s+i);
		}
		\end{lstlisting}
	
		Nach Kopie der Daten kann der Kommunikator initialisiert werden. In diesem Fall werden die Nummern der Geräte in \li`devs` in das Kommunikator-Array eingetragen. Folgende kollektiven Operationen stehen zur Verfügung:
	
		\begin{itemize}
		\item \textbf{\textit{All Reduce}}: Eine Reduktion, bei der das Ergebnis in jedem Element des Outputarrays bereit steht.
	
		\item \textbf{\textit{Broadcast}}: Kopiert ein Array eines Geräts auf alle anderen.
	
		\item \textbf{\textit{Reduce}}: Wie \textit{All Reduce}, aber das Ergebnis beschränkt sich auf den Outputbuffer eines einzigen Geräts.
	
		\item \textbf{\textit{All Gather}}: Auf jedem Gerät wird die selbe Anzahl an Werten von jedem Gerät aggregiert. Der Output wird nach Index des Geräts geordnet.
	
		\item \textbf{\textit{Reduce Scatter}}: Wie \textit{Reduce}, aber das Ergebnis wird in Blöcken auf alle Geräte verteilt. Jedes Gerät erhält einen Teil der Daten basierend auf dessen Index.
		\end{itemize}
	
		\begin{lstlisting}[caption=NCCL: Multi-Device Reduktion]
		ncclCommInitAll(comms, nDev, devs);

		ncclGroupStart();
		for(int i = 0; i < nDev; ++i)
		{
			ncclAllReduce((const void*)sendbuff[i], (void*)recvbuff[i], size, 
				ncclFloat, ncclSum, comms[i], s[i]);
		}
		ncclGroupEnd();
		\end{lstlisting}
	
		In diesem konkreten Beispiel wird für jedes Device die Maximumsreduktion auf den jeweiligen Buffern ausgeführt. Nach dem Aufruf von \li`ncclGroupEnd` steht das Gesamtergebnis in jedem Element der Outputbuffer bereit. Anschließend wird jedes Gerät synchronisiert.
		
		\newpage
	 
		\begin{lstlisting}[caption=NCCL: Synchronisation und Cleanup]
		for(int i = 0; i < nDev; ++i) 
		{
			cudaSetDevice(i);
			cudaStreamSynchronize(s[i]);
		}

		for(int i = 0; i < nDev; ++i) 
		{
			cudaSetDevice(i);
			cudaFree(sendbuff[i]);
			cudaFree(recvbuff[i]);
		}

		for(int i = 0; i < nDev; ++i)
			ncclCommDestroy(comms[i]);
		\end{lstlisting}    
	
		Neben der Freigabe der Buffer müssen zusätzlich die Kommunikatoren zerstört werden. Eine genauere Auflistung der Operationen sowie die Handhabung der Kommunikatoren kann im Handbuch {\small \url{https://docs.nvidia.com/deeplearning/sdk/nccl-developer-guide/docs/index.html}}\\ nachgelesen werden.
	
		\subsection{Device Abfrage mit MPI}
		Die genannten Operationen können nur auf maximal vier Geräte angewandt werden. Cluster bestehen jedoch aus hunderten GPUs. Typischerweise werden jeweils vier GPUs zusammengefasst und von einem Prozessor verwaltet. Zur Kommunikation dieser Prozessoren wir dann das sogenannte \textit{Message Passing Interface} (MPI) angewandt. Dabei handelt es sich um eine ganz eigene Art der Parallelisierung und soll daher hier nicht tiefer behandelt werden. Im Prinzip führt man dasselbe Programm auf allen Prozessoren unter Angabe von Hostnamen oder Nodes aus. In diesem Programm wird nach Initialisierung jedem Prozess eine Nummer (\enquote{Rank}) zugeordnet. Der nachfolgende Code wir dann von jedem Prozessor exakt zeitgleich ausgeführt. So kann jeder Prozessor seine Devices nacheinander unabhängig von den anderen abfragen.
		
		\newpage
		
		\begin{lstlisting}[caption=Device Abfrage mit MPI]
		MPI_Init(...);
   
		int world_size;
		MPI_Comm_size(MPI_COMM_WORLD, &world_size);

		int world_rank;
		MPI_Comm_rank(MPI_COMM_WORLD, &world_rank);

		char processor_name[MPI_MAX_PROCESSOR_NAME];
		int name_len;
		MPI_Get_processor_name(processor_name, &name_len);

		int deviceCount;
		cudaGetDeviceCount(&deviceCount);
		cudaDeviceProp deviceProp;
		for(uint device = 0; device < deviceCount; ++device)
		{
			cudaGetDeviceProperties(&deviceProp, device);
			printf("Host %s with rank %d (of %d) has %d device(s). 
				Its device #%d is %s and has compute capability %d.%d.\n", 
				processor_name, world_rank, world_size, deviceCount, 
				device+1, deviceProp.name, deviceProp.major, deviceProp.minor);
		}
    
		MPI_finalize();
	    \end{lstlisting}
    
    	Über den Rank kann ein Prozessor eine einzelne Aufgabe erfüllen, die nur für ihn bestimmt ist. Beispielsweise können zwei Prozesse direkt Daten miteinander austauschen. Mit einem GPU-Build von MPI ist es sogar möglich, explizit GPU-Memory zu senden. Dazu erstellt ein Prozess (0) einen Buffer für die Daten auf der Grafikkarte. Der andere Prozess (1) erstellt einen Buffer der selben Größe im Speicher seiner Grafikkarte. Prozess 0 kopiert die Daten. Dann sendet er explizit diese Daten an Rank 1. Rank 1 muss diese explizit abnehmen und speichert diese dabei in seinem Buffer ab. Zum Schluss können die Daten auf den neuen Host kopiert werden.
    	
    	\newpage
    	
	    \begin{lstlisting}[caption=Austausch von Device Memory mit MPI] 
    	MPI_Init(...);
    
    	int rank;   
	    MPI_Comm_rank(MPI_COMM_WORLD, &rank);
    
    	int size = ...;
	    if(rank == 0) { int *send_buf; cudaMalloc(&send_buf, size*sizeof(int)); }
    	if(rank == 1) { int *recv_buf; cudaMalloc(&recv_buf, size*sizeof(int)); }
    
    	...
    
    	if(rank == 0) 
    		cudaMemcpy(send_buf, ..., size*sizeof(int), cudaMemcpyHostToDevice);   
	    MPI_Send(send_buf, size, MPI_INT, 1, 0, MPI_COMM_WORLD);
	    ...
    	MPI_Recv(recv_buf, size, MPI_INT, 0, 0, MPI_COMM_WORLD, MPI_STATUS_IGNORE);
    	if(rank == 1) 
    		cudaMemcpy(..., recv_buf, size*sizeof(int), cudaMemcpyDeviceToHost);
    		
    	...
    
	    MPI_finalize();
    	\end{lstlisting}
    	
    	\subsection{GPU-Cluster}
    	Auf einem GPU-Cluster werden typischerweise beide Methoden kombiniert. Ein einzelnes Programm wird auf beliebig vielen Hosts ausgeführt. Die einzelnen Prozesse können mittels MPI explizit kommunizieren. Jeder dieser Prozesse kann frei zwischen bis zu vier Grafikkarten wählen und parallele Algorithmen unter Zuhilfenahme von NCCL ausführen. MPI ermöglicht das freie Senden der Daten ohne zusätzliches Buffering im Hostmemory. Die GPUs können mit PCIe oder NVlink angebunden sein. Im letzteren Fall sogar direkt untereinander. Die Prozessoren sind in einem Highspeed-Netzwerk (z.B. mit Infiniband) miteinander verbunden. Da diese Prozessoren normalerweise über mehrere Kerne verfügen, können diese, während die GPUs beschäftigt sind, mittels thread-parallelem Programmieren zusätzlich Aufgaben erledigen. So kann die Hardware vollständig ausgelastet werden.
    	
    	Das NCCL-Handbuch gibt unter\\ 
    	\url{https://docs.nvidia.com/deeplearning/sdk/nccl-developer-guide/docs/mpi.html} \\
    	Empfehlungen ab zur Verwendung von MPI mit NCCL. Ein Beispiel dafür ist unter \\
    	\url{https://docs.nvidia.com/deeplearning/sdk/nccl-developer-guide/docs/examples.html#example-3-multiple-devices-per-thread} \\
    	zu finden.
	\stopcontents[Cluster Computing mit NCCL]
    
    \newpage
		\section{Graph-Computing mit nvGRAPH}
	\url{https://docs.nvidia.com/cuda/nvgraph/index.html}