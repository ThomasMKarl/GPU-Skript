	\section{THRUST: STL f\"ur GPUs}
	THRUST ist die CUDA-Implementierung der Standard Template Library für C++ und ist Teil des CUDA-Toolkits. THRUST orientiert sich an C++11 und beseht aus vier Teilen.
		
		\subsection{Container}
		Grundlage für THRUST bilden die Container-Klassen zur Erzeugung dynamischer Datenstrukturen. Diese Klassen befinden sich im Namespace \li`thrust`. Die Klasse \li`host_vector` entspricht dabei dem C++ Vektor. Mit \li`device_vector` existiert eine Vektorklasse, die sich ebenfalls über Member-Funktionen steuern lässt, aber zu jedem Zeitpunkt im Speicher des Device liegt.
		\begin{lstlisting}[caption=THRUST Vektoren]
		#include <thrust/host_vector.h>
		#include <thrust/device_vector.h>

		...

		thrust::host_vector<int> H(4);

		H[0] = 14;
		H[1] = 20;
		H[2] = 38;
		H[3] = 46;

		std::cout << "H has size " << H.size() << std::endl;

		H.resize(2);

		thrust::device_vector<int> D = H;

		D[0] = 99;
		D[1] = 88;
		\end{lstlisting}
		
		In diesem Beispiel wird ein Vektor von Ganzzahlen erstellt und dessen Elemente gesetzt. Die Größe kann über die getter-Funktion \li`size` ausgegeben werden. Die Größe kann dynamisch verringert oder vergrößert werden. Danach wird der Hostvektor explizit in einen Devicevektor kopiert. Der Devicevektor lässt sich dabei auch vom Host modifizieren.
		
		Die Idee hinter Devicevektoren ist, Member-Funktionen zu benutzen, die auf diesen Datenstrukturen operieren und dabei in CUDA parallelisiert sind. Devicevektoren werden wie gewöhnliche Vektoren vom Host gesteuert, und können nicht direkt im Device Code verwendet werden, da sich im Device Speicher nicht dynamisch verwalten lässt.
		
		\subsection{Iteratoren}
		Zur besseren Steuerung der Vektoren bietet THRUST Iteratoren an.
		\begin{lstlisting}[caption=THRUST Iteratoren]
		thrust::device_vector<int> D(10, 1);
		thrust::fill(D.begin(), D.begin() + 7, 9);
		
		thrust::host_vector<int> H(D.begin(), D.begin()+5);
		thrust::sequence(H.begin(), H.end());
		
		thrust::copy(H.begin(), H.end(), D.begin());
		\end{lstlisting}
		
		\li`begin` und \li`end` sind Iteratoren, die auf Anfang und Ende des belegten Speichers zeigen. So können THRUST Funktionen benutzt werden, um Vektoren zu füllen oder zu Kopieren. Die erste Funktion füllt einen Vektor von Element 0 bis 6	mit 9. Dann wird ein Vektor erstellt als Kopie der ersten fünf Elemente des vorherigen. \li`sequenze` füllt den Vektor mit aufsteigenden Ganzzahlen. Im letzten Schritt wird ein Vektor von Anfang bis Ende An den Beginn eines anderen Vektors kopiert.
		
		Weitere Iteratoren sind
		\begin{itemize}
		\item \li`constant_iterator`
		\item \li`counting_iterator`
		\item \li`transform_iterator`
		\item \li`permutation_iterator`
		\item \li`zip_iterator`
		\end{itemize}
		und können in Kapitel 4 des Qick Start Guides nachgeschlagen werden. \autocite{thrustQSG}
		
		\subsection{Funktoren}\label{funk}
		Funktoren sind spezielle Datentypen, die Funktionen beinhalten, welche bestimmte Operationen ausführen. Diese Operationen sollen Elementweise auf einen Vektor angewendet werden. Folgendes Beispiel definiert einen Funktor für die Saxpy Operation für Host und Device:
		\begin{lstlisting}[caption=THRUST Funktoren]
		struct saxpy_functor
		{
			const float a;

			saxpy_functor(float _a) : a(_a) {}

			__host__ __device__
			float operator()(const float& x, const float& y) const {return a * x + y;}
		};
		\end{lstlisting}
		
		Nun kann mittels \li`thrust::transform(X.begin(), X.end(), Y.begin(), Y.begin(), saxpy_functor(A))` diese Operation auf jedes Element von \li`X` und \li`Y` angewendet und in \li`Y` gespeichert werden (siehe \ref{algo}).	Es existieren vordefinierte Funktoren wie \li`multiplies` oder \li`plus`.
		
		\subsection{Algorithmen}\label{algo}
		THRUST implementiert schnelle parallele Algorithmen in CUDA für dessen Containerklassen. Neben den genannten Transformationen (siehe \ref{funk}) existiert eine Implementierung eines parallelen Radix-Sortierverfahrens.
		\begin{lstlisting}[caption=THRUST Sortieren]	
		#include <thrust/sort.h>
		#include <thrust/functional.h>
		
		...
		
		const int N = 6;
		
		int A[N] = {1, 4, 2, 8, 5, 7};
		thrust::sort(A, A + N);
		thrust::sort(A, A + N, thrust::greater<int>());


		int    keys[N] = {  1,   4,   2,   8,   5,   7};
		char values[N] = {'a', 'b', 'c', 'd', 'e', 'f'};
		thrust::sort_by_key(keys, keys + N, values);
		\end{lstlisting}
		
		\li`sort` sortiert einen Vektor numerisch in aufsteigender Reihenfolge (inplace). Dies lässt sich mit einem Funktor kombinieren, um z.B. absteigend zu sortieren. \li`sort_by_keys` sortiert ein Array anhand von Eigenschaften eines anderen Arrays, den sogenannten Keys. In diesem Fall wird jedem Buchstaben eine Zahl zugeordnet und die Liste nach diesen Zahlen sortiert. Es existieren ebenfalls \li`stable_sort` uns \li`stabel_sort_by_keys`, die die relative Ordnung von Elementen mit gleichen Werten, die zur Ordnung klassifizieren, erhalten. Wenn also a und b den gleichen Key erhalten, ist so garantiert, dass a und b nicht vertauscht werden.
		
		\newpage
		Extrem nützlich für GPUs sind die Reduktionen von THRUST.
		\begin{lstlisting}[caption=THRUST Reduktionen]
		int sum = thrust::reduce(D.begin(), D.end(), (int) 0, thrust::plus<int>());
		
		#include <thrust/count.h>
		int result = thrust::count(vec.begin(), vec.end(), 1);
		\end{lstlisting}
		
		Erstere Funktion vollzieht eine Summenreduktion, Letztere zählt den Wert eins in einem Vektor. Zum Suchen bestimmter Werte implementiert THUST eine parallele Version einer Binärsuche. Eine Liste aller Reduktionen befindet sich unter \url{http://thrust.github.io/doc/group__reductions.html} in der THRUST Dokumentation. \autocite{thrustDoc}
		
		Mittels \li`transform_reduce` lassen sich Reduktionen mit selbstgeschriebenen Funktoren kombinieren. Soll eine Reduktion von THRUST auf ein Array angewendet werden, dass von einem \Gls{Kernel} modifiziert wurde oder an eines übergeben werden soll, eignet sich die Klasse \li`device_ptr`.		
		\begin{lstlisting}[caption=THRUST Device Pointer]
		size_t N = ...;

		int *raw_ptr;
		cudaMalloc((void **) &raw_ptr, N * sizeof(int))

		thrust::device_ptr<int> dev_ptr(raw_ptr);
		thrust::fill(dev_ptr, dev_ptr + N, (int) 0);

		thrust::device_ptr<int> dev_ptr = thrust::device_malloc<int>(N);
		raw_ptr = thrust::raw_pointer_cast(dev_ptr);
		\end{lstlisting}	
			
		Im ersten Beispiel wird aus einem gewöhnlichen Devicevektor ein \li`device_ptr` Objekt erstellt, im zweiten ein Pointer extrahiert. Auf dieses Objekt kann ein gewöhnlicher THRUST Algorithmus angewendet werden.
		
		Ferner existieren Scan\footnote{\url{http://thrust.github.io/doc/group__prefixsums.html}}- und Reordering\footnote{\url{http://thrust.github.io/doc/group__reordering.html}} Algorithmen.