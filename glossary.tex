\newglossaryentry{API}
{
	name=API,
	description={application programming interface, Programmierschnittstelle.},
	plural=APIs
}

\newglossaryentry{Arbeit}
{
	name=Arbeit,
	description={typischerweise eine Zahl von Flops, ein Maß für die Lesitung, die ein Programm erbringen muss.},
	plural=Arbeiten
}

\newglossaryentry{arithmetische Intensität}
{
	name=arithmetische Intensität,
	description={Verhältnis von Arbeit zum Datenverkehr.},
	plural=arithmetische Intensitäten
}

\newglossaryentry{Block}
{
	name=Block,
	description={ein evtl. mehrdimensionaler Verbund von Threads in Software (CUDA). Jeder Block besteht aus der gleichen Anzahl an Threads.},
	plural=Blöcke
}

\newglossaryentry{compute capability}
{
	name=compute capability,
	description={eine Versionsnummer der Chiparchitektur, die u.A. bestimmt, welche Features von CUDA auf der GPU implementiert wurden.},
	plural=compute capabilitys
}

\newglossaryentry{Command Queue}
{
	name=Command Queue,
	description={eine Menge von Instruktionen, die der Reihe nach abgehandelt werden sollen (OpenCL).},
	plural=Command Queues
}

\newglossaryentry{constant Memory}
{
	name=constant Memory,
	description={siehe erstes Vorkommen},
	plural=constant Memorys
}

\newglossaryentry{Datenverkehr}
{
	name=Datenverkehr,
	description={eine Zahl an Bytes, die während des Ausführens eines Programms insgesamt transferriert wird.}
}

\newglossaryentry{Gang}
{
	name=Gang,
	description={ein evtl. mehrdimensionaler Verbund von Vectors in Software (OpenACC). Jede Gang besteht aus der gleichen Anzahl an Vectors.},
	plural=Gangs
}

\newglossaryentry{global Memory}
{
	name=global Memory,
	description={siehe erstes Vorkommen},
	plural=global Memorys
}

\newglossaryentry{Grid}
{
	name=Grid,
	description={die Gesamtheit aller Blöcke (Nvidia).},
	plural=Grids
}

\newglossaryentry{Halfwarp}
{
	name=Halfwarp,
	description={ein Verbund von 16 Threads in Hardware (Nvidia, CUDA Terminologie).},
	plural=Halfwarps
}

\newglossaryentry{Handle}
{
	name=Handle,
	description={ein eindeutiger Referenzwert zu einer vom Betriebssystem verwalteten Systemressource.},
	plural=Handles
}

\newglossaryentry{Kernel}
{
	name=Kernel,
	description={eine kleine Recheneinheit, die auf die GPU ausgelagert werden soll.},
	plural=Kernel
}

\newglossaryentry{Kontext}
{
	name=Kontext,
	description={ein Handle für das OpenCL Programm (Speicherobjekte, Programs, Command Queues, ...).},
	plural=Kontexte
}

\newglossaryentry{local Memory}
{
	name=local Memory,
	description={Speicher (cached) eines einzelnen Threads (CUDA). \\ ein kleiner, sehr schneller Speicher, den sich alle Workitems einer Workgroup teilen (OpenCL). Pro SM wird einer verbaut (Nvidia).},
	plural=local Memorys
}

\newglossaryentry{MTIU}
{
	name=MTIU,
	description={Multi Threaded Instruction Unit, ein kleiner Controller, der Instruktionen an die Warps ausgibt.},
	plural=MTIUs
}

\newglossaryentry{nvcc}
{
	name=nvcc,
	description={Nvidia C(++) Compiler}
}

\newglossaryentry{nvlink}
{
	name=nvlink,
	description={eine High-Speed Datenschnittstelle für Nvidia Grafikkarten in Clustern.}
}

\newglossaryentry{NVPTX}
{
	name=NVPTX,
	description={Nvidia Parallel Thread Execution, eine assemblerartige Sprache, in die nvcc Devicecode übersetzt.}
}

\newglossaryentry{page-locked Memory}
{
	name=page-locked Memory,
	description={allozierter Speicher, der nicht vom Betriebssystem verschoben werden darf.}
}

\newglossaryentry{parallele Effizienz}
{
	name=parallele Effizienz,
	description={Verhältnis von Speedup zur Anzahl der eingesetzten Prozessoren.},
	plural=parallele Effizienzen
}

\newglossaryentry{PCIe}
{
	name=PCIe,
	description={Peripheral Component Interconnect Express, die standard-Schnittstelle für Steckkarten.}
}

\newglossaryentry{Peak Bandwidth}
{
	name=Peak Bandwidth,
	description={maximale interne Speicherbandbreite.},
	plural=Peak Bandwidth
}

\newglossaryentry{Peak Performance}
{
	name=Peak Performance,
	description={maximale Performance eines Prozessors.},
	plural=Peak Performances
}

\newglossaryentry{Performance}
{
	name=Performance,
	description={das Verhältnis von Flops zur benötigten Rechenzeit, ein Maß für die Effizienz von Soft- und Hardware.},
	plural=Performances
}

\newglossaryentry{Platform}
{
	name=Platform,
	description={eine OpenCL Imolementierung in Software.},
	plural=Platformen
}

\newglossaryentry{private Memory}
{
	name=private Memory,
	description={Speicher (cached) eines einzelnen Workitems (OpenCL).},
	plural=private Memorys
}

\newglossaryentry{shared Memory}
{
	name=shared Memory,
	description={siehe erstes Vorkommen},
	plural=shared Memorys
}

\newglossaryentry{SM}
{
	name=SM,
	description={Streaming Multiprocessor, Eine Recheneinheit bestehend aus mehreren Warps, einer MTIU und shared memory.},
	plural=SMs
}

\newglossaryentry{Speedup}
{
	name=Speedup,
	description={Faktor, um den ein paralleles Programm schneller läuft, als die sequentielle Variante.},
	plural=Speedups
}

\newglossaryentry{Stream}
{
	name=Stream,
	description={eine Menge von Instruktionen, die der Reihe nach abgehandelt werden sollen (CUDA).},
	plural=Streams
}

\newglossaryentry{texture Memory}
{
	name=Texture Memory,
	description={siehe erstes Vorkommen},
	plural=Texture Memorys
}

\newglossaryentry{Thread}
{
	name=Thread,
	description={die kleinste Recheneinheit einer GPU in CUDA, in Hardware auf Nvidia GPUs auch CUDA Core genannt.},
	plural=Threads
}

\newglossaryentry{Vector}
{
	name=Vector,
	description={die kleinste Recheneinheit einer GPU in OpenACC.},
	plural=Vectors
}

\newglossaryentry{Warp}
{
	name=Warp,
	description={Kombination zweier Halfwarps.},
	plural=Warps
}

\newglossaryentry{Wavefront}
{
	name=Wavefront,
	description={ein Verbund von 32 oder 64 Workitems in Hardware (AMD).},
	plural=Wavefronts
}

\newglossaryentry{Worker}
{
	name=Wavefront,
	description={ein Verbund von 32 Vectors in Hardware (Nvidia, OpenACC Terminologie).},
	plural=Workers
}

\newglossaryentry{Workgroup}
{
	name=Workgroup,
	description={ein evtl. mehrdimensionaler Verbund von Workitems in Software (OpenCL). Jede Workgroup besteht aus der gleichen Anzahl an Workitems.},
	plural=Workgroups
}

\newglossaryentry{Workitem}
{
	name=Workitem,
	description={die kleinste Recheneinheit einer GPU in OpenCL.},
	plural=Workitems
}
